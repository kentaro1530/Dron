\section{J-PARC E31 experiment}
As the situation is described in the previous section, it is desirable to measure $\bar{K}N$ directly scattering amplitudes, especially below the $\bar{K}N$ threshold.
However, due to energy conservation laws, kaon and nucleon cannot be directly scattered in free space.
Therefore, an experiment using the $d(K^- n)$ reaction was planned and carried out at J-PARC E31 \cite{E31_proposal}.
A similar experiment was carried out at CERN in 1977 using a bubble chamber and reported a spectrum with a peak position shifted to the high mass side above 1405 MeV,albeit only in $\Sigma^- \pi^+$
\cite{Braun}.
This spectrum shape was successfully reproduced by theoretical calculations using the chiral unitary model \cite{Jido2}.
It is known that $P-$wave scattering $\Sigma(1385)$ of isospin $I=1$ exists in the near region.
The $\pi \Lambda$ spectrum of $I=1$ in the same experiment was successfully reproduced by a similar theoretical calculation including $P-$waves scattering \cite{Yamagata}.
These theoretical calculations assume a 2-step reaction, $K^- p \rightarrow \bar{K}N$ scattering in 1 step and $\bar{K}N \rightarrow \pi \Sigma$ in 2 step.

In this experiment, due to the low momentum of the $K^-$ beam and the unknown angle of the nucleon knocked out in the first step,
some argued that there was a contribution from a reaction in which the $K^-$ beam and nucleon in deuteron were directly converted to $\pi\Sigma$ in the first step reaction
and the $\Lambda(1405)$ contribution was unknown.

Therefore, we measured the nucleon knocked out at super-forward angle employing the $d(K^-, n)$ reaction with 1GeV$/c$ $K^-$ beam by the forward detector systems.
In the case of a direct reaction between the $K^-$ beam and the nucleon to $\pi \Sigma$,
the $\pi \Sigma$ mass is distributed near the kinematic limit ($\sim 1.9$GeV$/c$) due to the small Fermi momentum of the nucleon in the deuteron and the energy given by the $K^-$ beam.
This means that the contribution from a direct 1-step reaction is negligible.
In the case of 2-step reaction, the $\bar{K}N \rightarrow \pi \Sigma$ scattering of the second step can be accessed to below the $\bar{K}N$ threshold
due to the virtual particle of recoiled $\bar{K}$ between the first and second steps.
In addition, the momentum transfer is small due to the small momentum of the recoiled $\bar{K}$,
and the second scattering is expected to enhance $S-$wave scattering against $P-$wave, $D-$waves and so on.

Also, previous experiment reported about $\pi^+ \Sigma^-$ and $\pi^- \Lambda$ spectra for $I=0$ and $I=1$, respectively.
Hence, it was not possible to decompose the isospin for these spectra and discuss the contribution of each isospin, in particular the contribution of the $I=0$ and $I=1$ interference term.
We identified the final state and decay modes by measuring decay particles from produced hyperons by the cryndrical detectors system srounding the liquid deuterium target
and performed isospin decomposition on the obtained spectra.

Since the J-PARC E31 experiment was proposed, theoretical $\pi\Sigma$ spectra were reported using various KN interactions and kinematics.
Onishi et al. reported the $\pi^-\Sigma^+, \pi^0 \Sigma^0$ and $\pi^+\Sigma^-$ spectrum from a full three-body calculation using two types of$\bar{K}N$N interactions,
whose one is an effective theory of ${\bf SU}(3)$ fields, called the energy-dependent model, and the other is a phenomenological potential, called the energy-independent model \cite{Ohnishi}.
Miyagawa et al. reported spectra contracted from  two subsystems, $\bar{K}N \rightarrow \bar{K}N$ and KN $\bar{K}N \rightarrow \pi \Sigma$.
The first step $\bar{K}N \rightarrow \bar{K}N$ interaction was used on the basis of recent partial wave analysis as it has a large $K^-$ beam momentum.
On the other hand, the second step $\bar{K}N \rightarrow \pi \Sigma$ interaction was used based on various results from the chiral unitary model, a low-energy scattering theory.


