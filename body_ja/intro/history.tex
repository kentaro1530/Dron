\section{History of the $\Lambda(1405)$}
$\Lambda(1405)$はストレンジネス$S=-1$、アイソスピン$I=0$、スピンとスピンパリティ$J^P=(1/2)^-$を含むハイペロンである。
最新の粒子データグループ(PDG) \cite{PDG}では、後述のいくつかの論文\cite{Dalitz}に基づいて、$\Lambda(105)$の質量と幅はそれぞれ $1405.1^{+1.3}_{-1.0}$MeV と$50.5\pm 2.0$MeV に割り当てられています。


$\Lambda(1405)$の存在は、1959年にDalitzとTaunによって$\bar{K}N$の準結合状態として初めて予言されました\cite{Dalitz_1st}。
1961年、ローレンス放射線研究所において、$K^-p\rightarrow \Sigma \pi \pi \pi$反応を用いた$\Sigma \pi$スペクトルで$\Lambda(1405)$的過剰状態がバブルチャンバーにより観測された\cite{L1405_LRL}。
彼らは、$\pi^- \Sigma^-$や$\pi^+ \Sigma^+$スペクトルの二重荷電スペクトルに対して、中性$\pi\Sigma$スペクトルで$\Lambda(1405)$的な過剰状態を報告した。
Hemingwayは,$4.2$GeVの$K^-$ビームを用いた水素バブルチェンバーによる$\Lambda(1405)$)の高統計生成に成功したことを報告した[4]。
彼らは、$K^- p \rightarrow \pi \Sigma(1660) \rightarrow \pi \pi \Lambda(1405) \rightarrow \pi \pi (\pi \Sigma)$の反応レンマの同定が$\Lambda(1405)$の生成を促進したと主張している。
DalitzとDeloffはこのデータで$\Lambda(1520)$と非共鳴からのからのバックグラウンドがないと予想される$\pi^+\Sigma^-$スペクトルにM-matrix/K-matrix解析を適用し、
$\Lambda(1405)$の質量と幅をそれぞれ$1406.4 \pm 4.0$MeV and $50\pm 2$MeVと評価した\cite{Dalitz}.


In the 2000's, the chiral unitary model claimed that the $\Lambda(1405)$ is a dynamical molecular state contributed from two poles,
$\pi\Sigma$ in the low-mass region and $\bar{K}N$ in the high-mass region.
According to the this model, the high-mass pole coupled to $\bar{K}N$ is $1426+16$MeV and the low-mass pole coupled to $\pi\Sigma$ is $1390+66i$MeV.
That means the $\pi \Sigma$ spectrum is expected to shift high mass region from the conventional $1405$MeV by directly accessing to the $\bar{K}N$ pole.

Experimentally, the $\Lambda(1405)$ production was also carried out employing various reaction mechanisms.
Niiyama et el. performed photoproduction $\gamma p \rightarrow K^+ \Lambda(1405)$ employing a $\gamma$ beam at $E_{\gamma}=1.5-2.4$GeV at the LEPS beamline in the Spring-8\cite{Niiyama}.
They measured the scattering angle in center of mass system of $K^+$ at $0.8<\Theta_{K^+}<1.0$,
reported mass spectra of $\pi^-\Sigma^+$ and $\pi^+\Sigma^-$ in the $\Lambda(1405) \rightarrow \pi \Sigma$ decay
and observed a difference between the two spectra in the $\Lambda(1405)$ region.
This fact means existance of the interference term between the isospin $I=0$ and $I=1$ channel.
The CLAS collaboration employing a $1.61$-$1.91$GeV $\gamma$ beam for photoproduction at the Jefferson Labolatory
and measured the scattering angle in center of mass system of $K^+$ at $0.6<\Theta_{K^+}<0.9$ \cite{CLAS,CLAS2}.
They reported all three $\pi^-\Sigma^+$, $\pi^+\Sigma^-$ and $\Sigma^0\pi^0$ spectra.
The centroid of those three spectra appear to be at $1405$MeV, but their lineshapes are different indicating contribution of $I=1$ strength in this reaction.

The HADES collaboration performed $\Lambda(1405)$ production in p-p collisions using the $3.5\mbox{GeV}/c$ proton beam \cite{HADES}.
They reported $\pi^-\Sigma^+$, $\pi^+\Sigma^-$ and these average spectra, which clearly show a peak below $1405$MeV.
