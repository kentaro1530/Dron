\begin{frame}{Content}
  \small
  \begin{itemize}
  \item \sout{ガウシアンでフィッティング}
    $\rightarrow$ 明らかに左右非対称成分がある
  \item \sout{MeanとRMSを使う方法}
    $\rightarrow$ ピークじゃないテール部分に引っ張られる
  \item 左右非対称なエラーを持つガウシアンのようなもの
    \begin{equation*}
      f_{fit}(x)=
      \left\{
      \begin{array}{ll}
        A\exp(\frac{\sigma_h^2}{(x-M)^2}) & (x>M) \\
        A\exp(\frac{\sigma_l^2}{(x-M)^2}) & (x<M)
        \end{array}
      \right.
    \end{equation*}
  \end{itemize}

  \begin{itemize}
  \item 全範囲でフィットするならまだテール成分に引っ張られている。\\
    $\rightarrow$ 範囲を区切る。どの範囲が適当か?\\
    ピークの高さの割合でフィット範囲を決める。\\
    全領域、$\exp(-3^2/2)\sim 3\sigma$相当...$\exp(-1/2)\sim 1\sigma$まで$0.5\sigma$刻み。
    % $\exp(-2.5^2/2)$、$\exp(-2^2/2)$、$\exp(1.5^2/2)$、$\exp(1/2)$。
    \sout{フィットする$\sigma$の範囲を変えてみる。\\
    フィットによって$M$も$\sigma_{h/l}$も変わるので範囲、$X \times \sigma$の$X$を決めて、\\繰り返し行い収束したところを使う。\\}
  \end{itemize}
\end{frame}
