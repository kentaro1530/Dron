Here, the spectra are decomposed into the $I=0$, $I=1$, and interference components in order to clarify the contriutions of these factors.
The obtained spectrum can be expressed as follows from the isospin relation.

\newcommand{\Tmat}{T_{\bar{K}N \rightarrow \pi \Sigma}}
\newcommand{\Cfirst}{C_{K^- N \rightarrow \bar{K}N}}
\newcommand{\CS}{\frac{d\sigma}{d\Omega dM}}
\begin{align}
  \CS(\pi^\mp \Sigma^\pm) \propto & \left| \Cfirst^0 \Tmat^{I=0} \mp \Cfirst^1 \Tmat^{I=1} \right|^2 \nonumber \\
  = & \left| \Cfirst^0 \Tmat^{I=0} \right|^2 + \left| \Cfirst^1 \Tmat^{I=1} \right|^2 \nonumber \\
    & \mp 2\mbox{Re}( \Cfirst^0 \Cfirst^1 \Tmat^{I=0} \Tmat^{I=1} ) & \label{eq:Charge_piS} \\
  \CS(\pi^- \Sigma^0) \propto & \left| \Cfirst^1 \Tmat^{I=1} \right|^2 \label{eq:pimS0}
\end{align}

Here, $\Tmat^{I=0,1}$ represents the $T$ matrix of the second $\bar{K}N \rightarrow \pi \Sigma$ scattering for isospin $I=0$ and $I=1$.
Additionally, $\Cfirst^{0,1}$ denotes the factor for the first $K^- p \rightarrow \bar{K}N$ scattering,
corresponding to the isospin $I=0$ and $I=1$ components of the second scattering.

Since it can be expressed as shown in Equation (\ref{eq:Charge_piS}), (\ref{eq:pimS0}),
the spectra corresponding to $I=0$, $I=1$, and their interference terms can be written as follows.
\begin{align}
   \CS(I=0) \propto & \frac{1}{2}\left( \CS(\pi^- \Sigma^+)+\CS(\pi^+ \Sigma^-)-\CS(\pi^-\Sigma^0) \right) \nonumber \\
            = & \left| \Cfirst^0 \Tmat^{I=0} \right|^2 \label{eq:I0}\\
   \CS(I=1) \propto & \CS(\pi^- \Sigma^0) \nonumber \\
            = & \left| \Cfirst^1 \Tmat^{I=1} \right|^2 \label{eq:I1}\\
   \CS(int) \propto & \left( \CS(\pi^- \Sigma^+)-\CS(\pi^+\Sigma^+) \right) \nonumber \\
            = & 4\mbox{Re}( \Cfirst^0 \Cfirst^2 \Tmat^{I=0} \Tmat^{I=1} ) \label{eq:interfer}
\end{align}

The spectrum is transformed according to this formula.
We introduce scaling parameters $S_{I=0}$ and $S_{I=1}$ to adjust the intensities of the $I=0$ and $I=1$ components,
respectively, so that the overall spectral shape becomes more comparable to the theoretical spectrum.
The interference terms are scaled by $\sqrt{S_{I=0}S_{I=1}}$, as described in Equation \ref{eq:interfer}.

\begin{figure}[htbp]
  \begin{tabular}{cc}
    \begin{minipage}{0.5\hsize}
      \centering
      \includegraphics[width=6.0cm]{../pic/Dron/fit_model_A_fix_Phi/I0_fit.eps}
    \end{minipage}

    \begin{minipage}{0.5\hsize}
      \centering
      \includegraphics[width=6.0cm]{../pic/Dron/fit_model_B_fix_Phi/I0_fit.eps}
    \end{minipage}
  \end{tabular}

  \begin{tabular}{cc}
    \begin{minipage}{0.5\hsize}
      \centering
      \includegraphics[width=6.0cm]{../pic/Dron/fit_model_A_fix_Phi/pimS0_fit.eps}
    \end{minipage}

    \begin{minipage}{0.5\hsize}
      \centering
      \includegraphics[width=6.0cm]{../pic/Dron/fit_model_B_fix_Phi/pimS0_fit.eps}
    \end{minipage}
  \end{tabular}

  \begin{tabular}{cc}
    \begin{minipage}{0.5\hsize}
      \centering
      \includegraphics[width=6.0cm]{../pic/Dron/fit_model_A_fix_Phi/interfer_fit.eps}
    \end{minipage}

    \begin{minipage}{0.5\hsize}
      \centering
      \includegraphics[width=6.0cm]{../pic/Dron/fit_model_B_fix_Phi/interfer_fit.eps}
    \end{minipage}
  \end{tabular}
  \caption{
    This figure illustrates the spectra of the experimental data, decomposed into $I=0$, $I=1$, their interference terms,
    and the corresponding fitting results obtained using Model A of the DCC model.
    The black error bars represent the experimental data, while the red line denotes the fit results.
    The upper-left panel corresponds to $I=0$, the upper-right panel to $I=1$, and the lower panel to the interference term.
  }
  \label{fig:fit_scale}
\end{figure}

\begin{table}[h]
  \caption{
    This table shows the pole position with $I=0$ and $J^P=1/2^-$ below the $\bar{K}N$ threshold by DCC models\cite{DCC2}
  }
  \centering
  \begin{tabular}{ccc}
    \hline
    \hline
    & pole1 & pole2 \\
    \hline
    Model.A & $1437-75i$ & $1372-56i$ \\
    Model.B & $1428-31i$ & $1397-98i$ \\
    \hline
    \hline
  \end{tabular}
  \label{table:DCC_poles}
\end{table}


By fitting with these parameters, we obtain $S_{I=0} = \fitAscaleIz$ and $S_{I=1} = \fitAscaleIo$ for Model A,
and $S_{I=0} = \fitBscaleIz$ and $S_{I=1} = \fitBscaleIo$ for Model B.
The results of these fits are shown in the figure \ref{fig:fit_scale} : the left panel corresponds to Model A, and the right panel to Model B.
The top subpanels show the spectra for the $I=0$ component, the middle ones for the $I=1$ component (i.e., the $\pi^-\Sigma^0$ spectrum),
and the bottom ones for the interference term.



\begin{figure}[htbp]
  \begin{tabular}{cc}
    \begin{minipage}{0.5\hsize}
      \centering
      \includegraphics[width=6.0cm]{../pic/Dron/fit_model_B_2/I0_fit.eps}
    \end{minipage}

    \begin{minipage}{0.5\hsize}
      \centering
      \includegraphics[width=6.0cm]{../pic/Dron/fit_model_B_2/pimS0_fit.eps}
    \end{minipage}
  \end{tabular}

  \centering
  \includegraphics[width=6.0cm]{../pic/Dron/fit_model_B_2/interfer_fit.eps}
  \caption{
    This figure shows the fitting results obtained when the degrees of freedom associated with the interference term
    are introduced as additional fitting parameters.
    The upper-left panel displays the $I=0$ component, the upper-right panel shows the $I=1$ component,
    and the lower panel presents the spectrum of the interference term.
    The error bars denote the experimental data, while the blue curve represents the fit obtained using DCC Model B.
    In this figure, the $I=1$ strength is determind by the only $\pi^- \Sigma^0$ spectrum.
  }
  \label{fig:fit_B_phase_fixI1}
\end{figure}

\begin{figure}[htbp]
  \begin{tabular}{cc}
    \begin{minipage}{0.5\hsize}
      \centering
      \includegraphics[width=6.0cm]{../pic/Dron/fit_model_B/I0_fit.eps}
    \end{minipage}

    \begin{minipage}{0.5\hsize}
      \centering
      \includegraphics[width=6.0cm]{../pic/Dron/fit_model_B/pimS0_fit.eps}
    \end{minipage}
  \end{tabular}

  \centering
  \includegraphics[width=6.0cm]{../pic/Dron/fit_model_B/interfer_fit.eps}
  \caption{
    This figure shows the results of the fitting using Model.B with the introduction of parameters related to the interference term.
    The notation is the same as in Fig.\ref{fig:fit_B_scale}.
    All three parameters are determined simultaneously in this fitting.
  }
  \label{fig:fit_B_phase}
\end{figure}


