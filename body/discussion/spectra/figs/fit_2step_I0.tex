\begin{figure}[htbp]
  \centering
  \begin{tabular}{ccc}
    \begin{minipage}{0.33\hsize}
      \includegraphics[width=4cm]{../pic/Dron/fit_I0_KzeroN.eps}
    \end{minipage}
    \begin{minipage}{0.33\hsize}
      \includegraphics[width=4cm]{../pic/Dron/fit_I0_KN.eps}
    \end{minipage}
    \begin{minipage}{0.33\hsize}
      \includegraphics[width=4cm]{../pic/Dron/fit_I0_Kmp.eps}
    \end{minipage}
  \end{tabular}
  \begin{tabular}{ccc}
    \begin{minipage}{0.33\hsize}
      \includegraphics[width=4cm]{../pic/Dron/fit_scat_amp_I0_KzeroN.eps}
    \end{minipage}
    \begin{minipage}{0.33\hsize}
      \includegraphics[width=4cm]{../pic/Dron/fit_scat_amp_I0_KN.eps}
    \end{minipage}
    \begin{minipage}{0.33\hsize}
      \includegraphics[width=4cm]{../pic/Dron/fit_scat_amp_I0_Kmp.eps}
    \end{minipage}
  \end{tabular}
  \caption{
    This figure shows the $I=0$ spectrum obtained from Eq. (\ref{eq:I0}) and the fit results assuming a 2-step reaction.
    In the upper panel, the error bars represent the experimental data, the red line indicates the spectrum resulting from the fit,
    the solid line corresponds to the spectrum convolved with detector resolution, and the dashed line represents the spectrum without resolution.
    The lower panel displays the scattering amplitude for the two-step $\bar{K}N \rightarrow \bar{K}N$ scattering,
    where the red line indicates the real part and the blue line shows the imaginary part.
    The lines are the best-fit values and the bands hatched in color represent the width of the error due to fitting.
  }
  \label{fig:fit_2step_I0}
\end{figure}
