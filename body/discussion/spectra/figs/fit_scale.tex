\begin{figure}[htbp]
  \begin{tabular}{cc}
    \begin{minipage}{0.5\hsize}
      \centering
      \includegraphics[width=6.0cm]{../pic/Dron/fit_model_A_fix_Phi/I0_fit.eps}
    \end{minipage}

    \begin{minipage}{0.5\hsize}
      \centering
      \includegraphics[width=6.0cm]{../pic/Dron/fit_model_B_fix_Phi/I0_fit.eps}
    \end{minipage}
  \end{tabular}

  \begin{tabular}{cc}
    \begin{minipage}{0.5\hsize}
      \centering
      \includegraphics[width=6.0cm]{../pic/Dron/fit_model_A_fix_Phi/pimS0_fit.eps}
    \end{minipage}

    \begin{minipage}{0.5\hsize}
      \centering
      \includegraphics[width=6.0cm]{../pic/Dron/fit_model_B_fix_Phi/pimS0_fit.eps}
    \end{minipage}
  \end{tabular}

  \begin{tabular}{cc}
    \begin{minipage}{0.5\hsize}
      \centering
      \includegraphics[width=6.0cm]{../pic/Dron/fit_model_A_fix_Phi/interfer_fit.eps}
    \end{minipage}

    \begin{minipage}{0.5\hsize}
      \centering
      \includegraphics[width=6.0cm]{../pic/Dron/fit_model_B_fix_Phi/interfer_fit.eps}
    \end{minipage}
  \end{tabular}
  \caption{
    This figure illustrates the spectra of the experimental data, decomposed into $I=0$, $I=1$, their interference terms,
    and the corresponding fitting results obtained using Model A of the DCC model.
    The black error bars represent the experimental data, while the red line denotes the fit results.
    The upper-left panel corresponds to $I=0$, the upper-right panel to $I=1$, and the lower panel to the interference term.
  }
  \label{fig:fit_scale}
\end{figure}
