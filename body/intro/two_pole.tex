A theoretical extrapolation of the $\bar{K}N$ scattering amplitude to the low-momentum regime was made in an effort to better understand the low-energy behavior.
However, applying a low-momentum perturbation expansion to the strangeness sector proved to be challenging due to the presence of $\Lambda(1405)$.

This model suggested that $\Lambda(1405)$ receives contributions from two poles \cite{Jido}:
one being a bound state located in the high-mass region of the $\bar{K}N$ system at $1426+16i$ MeV,
and the other being a resonant state of the $\pi \Sigma$ system located in the low-mass region at $1390+66i$ MeV.
Due to contributions from both poles,
the spectral shape and intensity of $\Lambda(1405)$ are expected to depend on the reaction mechanism and the $\pi \Sigma$ decay modes.
Therefore, it is important to collect data on various $\pi \Sigma$ modes through the use of different reaction mechanisms.
