Experiments using two reaction mechanisms were conducted in this context.
One is photoproduction, $\gamma p \rightarrow K^+ \Lambda(1405)$, and the other is proton-proton collisions, $p p \rightarrow n K^+ \Lambda(1405)$.

The photoproduction experiment was first performed at LEPS using a 1.5-2.4 GeV $\gamma$-ray beam,
where the $\pi^- \Sigma^+$ and $\pi^+ \Sigma^-$ spectra were reported \cite{Niiyama}.
The difference between these two spectra was attributed to the interference between the $I=0$ and $I=1$ components,
and this interference term was observed.
Subsequently, the CLAS Collaboration reported the spectra of all three $\pi\Sigma$ modes,
i.e., $\pi^- \Sigma^+$, $\pi^+ \Sigma^-$ and $\pi^0 \Sigma^0$, using 1.61-1.91 GeV $\gamma$-ray beams,
and all three spectral shapes were different \cite{CLAS}.
They also experimentally confirmed that $\Lambda(1405)$ has isospin $I=0$ and spin-parity $I^P=(\frac{1}{2}^-)$ \cite{CLAS2}.
Those three spectra were reproduced by theoretical calculations of the photoproduction process using the chiral unitary model \cite{CLAS_chU}.
However, in these calculations, some phenomenological parameters were adjusted by fitting the spectra,
indicating that the photoproduction process is complex and that it is not possible to conclude how the $\Lambda(1405)$ poles affect the spectra.

Results from proton-proton scattering were reported by the HADES collaboration using proton beam with 3.5 GeV,
which presented spectra of $\pi^- \Sigma^+$ and $\pi^+ \Sigma^-$ \cite{HADES}.
Differences were observed in these spectra, indicating that $I=0$ and $I=1$ interference was also present in proton-proton scattering.
The peaks of these two spectra are clearly located below 1400 MeV,
indicating that the peak position of $\Lambda(1405)$ strongly depends on the production mechanism.
However, as in the case of photoproduction, the production mechanism of $\Lambda(1405)$ in proton-proton scattering is complex,
and it is not possible to conclude how its poles affect the spectrum.

