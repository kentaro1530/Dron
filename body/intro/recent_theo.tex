\begin{table}
  \begin{tabular}{cccccccc}
    \hline
    & \multicolumn{3}{c}{Higher pole [MeV]} & \multicolumn{3}{c}{Lower pole [MeV]} \\
    \hline \hline
    J. Dido et el.\cite{Jido}           & $1426$            & $-i$ & $16$            & $1390$             & $-i$ & $66$ \\ 
    Y. Ikeda et el.\cite{Ikeda}          & $1424^{+7}_{-23}$ & $-i$ & $26^{+3}_{-14}$ & $1381^{+18}_{-6}$  & $-i$ & $81^{+19}_{-8}$ \\
    Z.-H. Geo et el.\cite{Geo}         & $1421^{+3}_{-2}$  & $-i$ & $19^{+8}_{-5} $ & $1388^{+9}_{-9} $  & $-i$ & $114^{+24}_{-25}$ \\
    M. Mai et el.\cite{Mai} Solution.2 & $1434^{+2}_{-2}$  & $-i$ & $10^{+2}_{-1} $ & $1330^{+4}_{-5} $  & $-i$ & $56^{+17}_{-11}$ \\
    M. Mai et el.\cite{Mai} Solution.4 & $1429^{+8}_{-7}$  & $-i$ & $12^{+2}_{-3} $ & $1325^{+15}_{-15}$ & $-i$ & $90^{+12}_{-18}$ \\
    DCC.A\cite{DCC1}                    & $1432$            & $-i$ & $75$            & $1372$             & $-i$ & $59$ \\
    DCC.B\cite{DCC1}                    & $1428$            & $-i$ & $31$            & $1397$             & $-i$ & $98$ \\
    \hline
  \end{tabular}
  \caption{
    This table shows the pole positions for $\Lambda(1405)$.
  }
  \label{ref:L1405_pole}
\end{table}


Although the various experiments described above do not provide conclusive information on the poles of $\Lambda(1405)$,
they play an important role in theoretical studies.
Building on these results, theoretical studies based on the chiral unitary model have aimed to obtain more detailed $\bar{K}N$ scattering amplitudes while imposing chiral symmetry and unitarity conditions.

Y. Ikeda et al. \cite{Ikeda} and Z.-H. Geo et al. \cite{Geo} theoretically calculated the $\bar{K}N$ scattering amplitudes
by incorporating not only the results of $\bar{K}N$ scattering experiments but also the SIDDARTA experimental results as constraints,
including not only the leading order but also the next-to-leading (NLO) order in their analysis.
M. Mai et al. derived several solutions through a similar analysis and tested them against the results of the CLAS experiment \cite{Mai}.
However, since the CLAS experimental data require reaction-specific calculations for fitting,
they could not be directly incorporated into the fitting procedure.
Instead, they were used as a filter to evaluate the resulting solutions, leaving two viable candidates.

Since chiral unitary calculations are based on perturbative expansions at low momentum,
low-momentum $\bar{K}N$ scattering data have been primarily used.
However, due to experimental difficulties, the available low-momentum data are insufficient,
whereas medium- to high-momentum scattering data are more abundant.
Partial wave analysis is known as a method applicable to this energy region and has been extensively studied,
as described in Section \ref{sec:KN_interaction}.
It continues to evolve with the incorporation of new data \cite{KSU,KSU2}.
However, even with this approach, drawing conclusions at low momentum has been difficult due to the presence of $\Lambda(1405)$.
To overcome this challenge, the $\bar{K}N$ scattering amplitudes analyzed using the DCC model, a dynamic extension of partial wave analysis, were reported \cite{DCC1}.
The analysis produced two solutions for the scattering amplitudes, primarily due to the lack of data in the low-momentum region.
Although both scattering amplitudes exhibit two poles in the $S$-wave scattering below the $\bar{K}N$ mass threshold,
similar to the chiral unitary model, the parameters of these poles were very different.
The pole positions of $\Lambda(1405)$ based on the theoretical analysis presented in this chapter are summarized in Figure \ref{ref:L1405_pole}.
