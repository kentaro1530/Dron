% \begin{table}
  \begin{tabular}{cccccccc}
    \hline
    & \multicolumn{3}{c}{Higher pole [MeV]} & \multicolumn{3}{c}{Lower pole [MeV]} \\
    \hline \hline
    J. Dido et el.\cite{Jido}           & $1426$            & $-i$ & $16$            & $1390$             & $-i$ & $66$ \\ 
    Y. Ikeda et el.\cite{Ikeda}          & $1424^{+7}_{-23}$ & $-i$ & $26^{+3}_{-14}$ & $1381^{+18}_{-6}$  & $-i$ & $81^{+19}_{-8}$ \\
    Z.-H. Geo et el.\cite{Geo}         & $1421^{+3}_{-2}$  & $-i$ & $19^{+8}_{-5} $ & $1388^{+9}_{-9} $  & $-i$ & $114^{+24}_{-25}$ \\
    M. Mai et el.\cite{Mai} Solution.2 & $1434^{+2}_{-2}$  & $-i$ & $10^{+2}_{-1} $ & $1330^{+4}_{-5} $  & $-i$ & $56^{+17}_{-11}$ \\
    M. Mai et el.\cite{Mai} Solution.4 & $1429^{+8}_{-7}$  & $-i$ & $12^{+2}_{-3} $ & $1325^{+15}_{-15}$ & $-i$ & $90^{+12}_{-18}$ \\
    DCC.A\cite{DCC1}                    & $1432$            & $-i$ & $75$            & $1372$             & $-i$ & $59$ \\
    DCC.B\cite{DCC1}                    & $1428$            & $-i$ & $31$            & $1397$             & $-i$ & $98$ \\
    \hline
  \end{tabular}
  \caption{
    This table shows the pole positions for $\Lambda(1405)$.
  }
  \label{ref:L1405_pole}
\end{table}

The J-PARC E31 experiment employs a two-step reaction, as shown in Figure \ref{fig:kd_diag}-(b).
In this reaction, the irradiated $K^-$ beam knocks out a nucleon in deuterium, which are then scattered forward.
In addition, a Cylindrical Detector System (CDS) is installed around the liquid deuterium target to identify the $\pi \Sigma$ decay mode,
and the particles originating from the decay of the produced $Y^*$ will be measured.
In this experiment, a 1 GeV/c $K^-$ beam is used, which frequently undergoes $K^-N \rightarrow \bar{K}N$ elastic scattering.
When the missing mass of $Y^*$ produced in this reaction is near the $\bar{K}N$ threshold,
the momentum of the scattered $\bar{K}$ is around 250 MeV$/c$, making it easier to induce a secondary reaction with the residual nucleon.

In this experiment, all $\pi \Sigma$ modes are measured, i.e., the $\pi^-\Sigma^+$, $\pi^0 \Sigma^0$, and $\pi^+ \Sigma^-$ spectra.
However, this paper focuses on discussing the $\pi^- \Sigma^+$ and $\pi^+ \Sigma^-$ spectra.
These spectra include contributions from $I=0$ and $I=1$ states, and their interference terms.
However, using only these two spectra, it is impossible to fully disentangle the individual contributions.
Therefore, we also discuss the $K^- d \rightarrow p \pi^- \Sigma^0$ spectrum, in which the proton is detected in the forward direction.
This spectrum is purely associated with $I=1$, and by incorporating this information,
the $I=0$, $I=1$, and interference component can be fully disentangled.

One such calculation was performed by Miyagawa et al \cite{Miyagawa}.
They used the scattering amplitudes from the latest partial-wave analysis for the first-step $\bar{K}N \rightarrow \bar{K}N$ scattering
and those from some chiral unitary approaches for the second-step $\bar{K}N \rightarrow \pi \Sigma$ scattering.
Thus, their calculation adopts a \textit{hybrid} method that combines two different approaches.
This is because the chiral unitary approach is specialized for low-energy interactions
and is not applicable to the first-step $\bar{K}N\rightarrow \bar{K}N$ scattering.
The spectrum of this experiment, considering the same two-step scattering process, has also been calculated using the DCC model.
As mentioned in Section \ref{sec:recent_theo},
this model extends the partial-wave analysis, which had been difficult to apply to the $\Lambda(1405)$ region,
thereby making it applicable to that region.

The setup for this experiment, briefly introduced at the beginning of this chapter, is detailed in Chapter \ref{chapter:setup}.
The calibration and analysis methods for the detectors are described in Chapter \ref{chapter:analysis}.
The selection of $\pi\Sigma$ modes from the measured data and their conversion to cross sections are discribed in Chapter
\ref{chapter:piSigma_spectra}.
The physical interpretation of the obtained spectra is discussed in Chapter \ref{chapter:discussion}.
There, we discuss how well $\bar{K}N$ scattering is understood by comparing it with the theoretical calculations presented in this chapter,
especially in the region of $\Lambda(1405)$.
In doing so, we examine the contribution of each component by decomposing it into $I=0$, $I=1$, and their interference terms.
Finally, using the $I=0$ spectrum obtained from this experiment, we determine its scattering length and effective range,
from which we derive the pole parameters of $\Lambda(1405)$, namely its mass and width.

