\chapter{Conclusion}
We have measured $\pi \Sigma$ masses using the missing mass method with nucleons scattered forward using a $K^-$ beam with $1 GeV/c$
and a liquid deuterium target at the K1.8BR beamline in the hadron hall of J-PARC.
In this reaction, the irradiated $K^-$ beam kicks out nucleons in deuterium and the decelerated $\bar{K}$ is scattered.
By measuring the forward nucleons in particular, the momentum of $\bar{K}$ in the laboratory system is lower, making it more likely to react with residual nucleons.
We used CDS to identify the $\pi \Sigma$ final state to  $\pi^- \Sigma^+$, $\pi^+ \Sigma^-$ and $\pi^0 \Sigma^0$ modes.

In this reaction, especially around the $\bar{K}N$ threshold,
we can expect $S-$waves to be enhanced in the second-step $\bar{K}N \rightarrow \pi \Sigma$ scattering due to the smaller angular momentum brought in by the mediating $\bar{K}$.
In fact, no structure is observed in the $\pi^- \Sigma^0$ spectrum at $I=1$ in the region of the known $P-$wave, $\Sigma(1385)$.
Also in the $\pi^{\mp} \Sigma^{\pm}$ spectrum with $I=0$ and $I=1$ contributions,
no structure is observed neither in $\Sigma(1385)$ nor in the region of $\Lambda(1520)$, known as the $D-$wave.
Therefor, it is confirmed that the $S-$wave is dominant in present reaction.

It is known that $S-$wave $\Lambda(1405)$ exists at $I=0$ just below the KN threshold, and its relation to the $\bar{K}N$ interaction,
which is a strong attraction, has long been discussed.
Although the elementary process of KN scattering cannot extract information below the KN threshold,
this reaction is a 2-step reaction and
by reacting the mediating $\bar{K}$ with the residual nucleon, $\bar{K}N \rightarrow \pi \Sigma$ scattering information below the $\bar{K}N$ threshold can be obtained.
The $\pi^- \Sigma^0$ spectrum with $I=1$ has a similar shape to the quasi-elastic scattering from the first-step $K^- N \rightarrow \bar{K}N$ scattering,
suggesting that there is no structure in the second-step $\bar{K}N \rightarrow \pi \Sigma$ scattering in this region,
while the $\pi^{\mp} \Sigma^{\pm}$ spectrum with $I=0$ shows an excess below $\bar{K}N$ threshold.
This means the presence of a $\Lambda(1405)$ contribution.
An interference term between $I=0$ and $I=1$, which appears as a difference in the $\pi^{\mp} \Sigma^{\pm}$ spectra, is also observed.

We decompose the measured spectra into $I=0$ and $I=1$ and interference term.
Since this experiment was proposed, several theory-based predicted spectra have been calculated.
We demonstrate with the respective strength as parameters using the DCC model Model.B which best reproduces our data.

When only the isospin $I=0$ and $I=1$ strength are parametezized,
it is find that the $I=0$ strength should be multiplied by a factor of $\fitBscaleIz$ and the $I=1$ strength by a factor of $\fitBscaleIo$.
The $\chi^2/NDF$ of this fit is $\fitBscaleChi=\fitBscaleChiNum$.
Furthermore, when the interference term is parameterized to vary independently,
we tried two methods: one to determine the intensity of $I=1$ only from the $\pi^- \Sigma^0$ spectrum, and the other to fit all the parameters at the same time.
These fittings of $\chi^2/NDF$ are $\fitBBChi=\fitBBChiNum$ and $\fitBChi=\fitBChiNum$, with values very close and improved compared to before the parameters were introduced.
If the difference between these two values is regarded as a systematic error,
it was found that the strength of $I=0$ needs to be multiplied by $A_{I=1}=1.516 \pm 0.054 \mbox{(Systematic)} ^{+0.058} _{-0.059} \mbox{(fitting)}$,
the strength of $I=1$ by $A_{int}=0.820 \pm 0.09 \mbox{(Systematic)} \pm 0.03 \mbox{(fitting)}$,
and the strength of the interference term by $34.9 \pm 0.9 \mbox{(Systematic)} \pm 0.3 \mbox{(fitting)}$.
Changing the interference term by this amount corresponds to shifting the phase difference between $I=0$ and $I=1$ by $34.9 \pm 0.9 \mbox{(Systematic)} \pm 0.3 \mbox{(fitting)}$ degrees.
As described, the spectra we obtained contains information on $\bar{K}N \rightarrow \pi \Sigma$ scattering including below the $\bar{K}N$ threshold as second scattering.
From that spectra we report experimental values for the $I=0$ strength as well as the $I=1$ strength and the phase difference between them.


% Furthermore, when the interference terms were parameterized to vary independently, we tried two methods: one to determine the intensity of I=1 only from the pi Sigma spectrum, and the other to fit all the parameters at the same time.

% In short, we measure the $\pi- \Sigma^0$, $\pi^+ \Sigma^-$ and $\pi^+ \Sigma^-$ spectra using the $d(K^-, N_{forward})"\pi \Sigma"$ reaction with a 1GeV$/c$ $K^-$ beam.
