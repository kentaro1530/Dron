\subsection{Generated data by Monte Calro simulation} \label{subsec:tempFit_MCdata}
\begin{eqnarray}
  & &  K^- d \rightarrow n K^0 n \label{reaction:prodK0} \\
  & &  K^- d \rightarrow \pi^\mp \Sigma^\pm n_{forward} \label{2step:piS}\\
  & &  K^- d \rightarrow \pi^\mp \Sigma^\pm n_{missing} \label{reaction:prodSigma}
\end{eqnarray}

Since the $K^- d\rightarrow n \pi^+ \pi^- n$ event is expected to contain 3 type reactions, (\ref{reaction:prodK0})-(\ref{2step:piS}),
we obtain the $d(K^-, n)"\pi^{\mp}\Sigma^{\pm}"$ event rejecting $K^0$ and $\Sigma^{\pm}_{forward}$ in $K^- d \rightarrow n \pi^+ \pi^- n$ event.

We perform template fitting using data reproduced using Monte Carlo simulation (geant4) to decompose into $\pi^-\Sigma^+$ and $\pi^+\Sigma^-$ modes.
In this subsection, we explain how we reproduce the data using geant4 simulation.

The $K^0$ production in (\ref{reaction:prodK0}) is simply the so-called quasi-elastic scattering,
in which an initial $K^-$ reacts with a proton and is converted into $K^0$ and a neutron, where the residual neutron is a spectator and its momentum is the Fermi momentum.
This reaction causes the scattering angles of the reacting protons and $K^-$ to be distributed in a way that reproduces the past experiment of one nucleon scattering \cite{KP_K0n},
and the momentum of the spectator is also distributed in a way that reproduces the past experiment \cite{d_fermi_ex}.
In the case of $\Sigma^{\pm}_{forward}$ production in (\ref{reaction:prodSigma}), the angular distribution of the past experiment\cite{KP_CEX_1GeV} is simulated in the same way.

In the $K^0$ produced event (\ref{reaction:prodK0}), in addition to 1-step reaction described above, events such as a 2-step and direct $\Lambda(1520)$ production are observed.
So, we generate data of these reaction.
In the case of 2-step reaction, the momentum of recoiled $\bar{K}$ is small and the scattering data of such $\bar{K}$ and nucleon are a few,
so the angular distribution and other details are not known.
Therefore, the MC data is generated assuming that the scattering of recoiled $\bar{K}$ and nucleons is isotropic. 
In the case of direct $\Lambda(1520)$ production also no data.
Therefore, this data was isotropically scattered $K^-d \rightarrow n \Lambda(1520)$ and $\Lambda(1520)$ decayed to $n$ and $K^0$.

Next, we explain the main signal (\ref{2step:piS}), which is the backward $\pi\Sigma$ generation.
Since there is no data on the invariant mass of $\pi\Sigma$, it is generated as a uniform distribution from the $\bar{K}N$ threshold, whose lineshape is determined by template fitting.
There is also no data on the scattering angle of $\pi\Sigma$,
but since the $\bar{K}N\rightarrow \pi\Sigma$ scattering is expected to be $S$-wave in this reaction,
we assume that it is isotropic and generate MC data.

In summary, the following seven MC data are used for template fitting.
\begin{itemize}
  \item About $K^0$ production reaction
  \begin{itemize}
  \item
    $ K^- d \rightarrow n_{forward} K^0 n_{spectator}$ \hspace{\fill} ($K^0$ 1-step)
  \item
    $K^- d \rightarrow n_{forward} (K^0 n)_{isotropic}$ \hspace{\fill} ($K^0$ 2-step)
  \item
    $K^- d \rightarrow n \Lambda(1520) \rightarrow n K^0 n$ \hspace{\fill} (direct $\Lambda(1520)$)
  \end{itemize}
\item About $\Sigma_{forward}$ production reaction
  \begin{itemize}
  \item
    $K^- d \rightarrow \Sigma^+ \pi^- n_{spectator} \rightarrow n_{forward} \pi^+ \pi^- n_{spectator}$ \hspace{\fill} ($\Sigma^+$ 1-step)
  \item
    $K^- d \rightarrow \Sigma^- \pi^+ n_{spectator} \rightarrow n_{forward} \pi^- \pi^+ n_{spectator}$ \hspace{\fill} ($\Sigma^-$ 1-step)
  \end{itemize}
\item About backward $\pi \Sigma$ production reaction
  \begin{itemize}
  \item
    $K^- d \rightarrow n_{forward} \pi^- \Sigma^+$ \hspace{\fill} (backward $\pi^- \Sigma^+$)
  \item
    $K^- d \rightarrow n_{forward} \pi^+ \Sigma^-$ \hspace{\fill} (backward $\pi^+ \Sigma^-$)
  \end{itemize}
\end{itemize}

