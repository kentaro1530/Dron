\subsection{Neutron Counter - NC}
The neutron counter (NC) are located 14.7 m upstream of the target.
Because the NC are located at the most upstream and the purpose is to detecting neutral particles, the NC requires a large amount of material.
Therefore, the NC is a segmented scintillation detector with 7 layers, each layer consisting of 16 counters.
One scintillation detector is 20cm (width) $\times$ 150cm (height) $\times$ 5cm (thickness) in size.
So one layer covers 320cm (width) $\times$ 150cm (height), which is corresponds 6.2 degrees in horizontal and 2.9 degrees in vertical in this experiment setup.
The first three layers of the scintillator are made of Saint-Gobain BC408 and the other four layers are made of Saint-Gobain BC412.
The scintillation light is carried by lucite light guides on both sides and read out by the 2-inch PMT (Hamamatsu H6410).
Each layer is installed with a gap of 2 cm, so the whole NC has a thickness of 47 cm.
Differences between the upstream and downstream solid angles are evaluated as systematic errors.
