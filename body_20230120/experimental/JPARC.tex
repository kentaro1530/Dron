\section{Experimental facility}
\subsection{J-PARC}

Our experiment was performed at the K1.8BR beamline at hadron facility of the J-PARC located at the Tokai site in Ibaraki Prefecture\cite{JPARC_had,K18BR}.
J-PARC, which means Japan Proton Accelerator Research Complex, consists of some facilities,
which are nuclear transmutation facility, materials, and life experimental facility, muon facility, neutrino facility, and hadron facility.
The concept of the J-PARC is to provide various secondary beam for the above purpose.
The J-PARC has three accelerators,
first one is linac which is injector and accelerates proton beam to $400MeV/c$,
second is RCS (Rapid Cycling Synchrotrons) which accelerates proton beam to $3GeV$, which was provided to materials and life experimental facility and muon facility.
Next is MR (Main Ring) which accelerates proton beam to $30GeV/c$, which beam was extracted by two methods.
One is the fast extraction (FX) for the neutrino facility to produce a neutrino beam which transported to the super Kamiokande.
The other is the slow extraction (SX) for the hadron facility.
In this extraction, banched beam in the MR is gradually extracted as scraping.
For this purpose, the $30 GeV/c$ proton beam was extracted about 2-second with a 5.2-second repetition cycle.

This continuous beam is irradiated on the primary target that is 6mm $\times$ 6mm $\times$ 66mm golden block
to generate secondary beam which includes anti-proton, pion, kaon and so on that is not naturally exist.
The secondary beam was transported to several beamlines.

The present experiment was performed at the K1.8BR beamline, which was placed at north of the hadron facility and branced from the K1.8 beamline
which was optimized for beam whose momentum is around $1.8 GeV/c$.


