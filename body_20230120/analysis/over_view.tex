\section{Over view}
This chapter describes data analysis.
First, the general analysis method for hodoscopes and drift chambers will be described.
Next, the analysis of beamlines is described. In this analysis, kaon is selected and accidental events, such as the passage of two particles, are eliminated to define the beam irradiated to the target.
The number of irradiated beams is calculated from the scaler and the luminosity is calculated using the beam survival rate DAQ efficiency and the trigger efficiency.

Next, the CDS is explained.
The CDC performed tracking of decayed particles in the solenoid magnet and decided reaction vertex with the beamline chamber which is called the BPC.
The CDH measured the timing of decayed particles and identified these particles with momentum reconstructed by CDC.
The efficiency of the CDC was measured using the CDH and the IH which were installed at the outer and the inner of the CDC as trigger counters.

Next, forward detectors were explained.
These were divided into the forward charge particles and forward neutral particles.
The beam was swept to the Beam Dump.
Forward scattered particles were detected by the Neutron Counter (NC), the Charge Veto Counter (CVC), and the Proton Counter (PC).
Positive charge particles were swept to the opposite direction of the beam, so the PC was placed about 15m downstream of this direction.
Although the purpose of the CVC is to veto for neutral particles, half of this counter was used detection of positive particles scattered forward to cover the high momentum region.
Because these particles were bent, the Forward Drift Chamber (FDC1) was installed between the target system and the Ushiwaka magnet.
The trajectory of the forward charge particle was determined using the reaction vertex point, the position before bent, and counter hit position.
The particle identification of a forward charged particle was performed using the momentum estimated from the bending angle and velocity calculated from the T0-PC TOF.
And, the momentum of forward scattered proton was finally calculated from the TOF method due to the high resolution.
The efficiency of the forward proton was evaluated using the $d(K^-, (\pi^-\Lambda)_{CDS})"p"$ reaction which was searched in events detecting 2 protons and $\pi^-$ by the CDS.
% In these events, the $\Lambda$  was reconstructed and the $d(K^-, \pi\Lambda)"p"$ was identified.
We selected the $d(K-, p)"\pi^-\Lambda"$ and the $d(K-, p)"\pi^-\Sigma^0$ final state from missing masses, which were corrected the acceptance that evaluated using the Monte Carlo simulation.
After that, were converted to the cross section using conversion factors that consist of the luminosity and detector efficiencies.

Because forward scattered neutral particles straightly go to the NC, the momentum of the forward neutral particle was measured by the TOF methold.
The flight length was determined from the counter position and the vertical position which was estimated from the time difference of the up and the down PMT.
The contamination from the charged particles were rejected by the BVC ant the CVC which were install just upstream of the Ushiwaka magnet and the NC.
We also adopt the energy deposit threshold for the rejection of accidental events which was estimated from the the S/N ratio and purity of the neutron at the Quasi-elastic region.
The efficiency of the NC system was decomposed to the two parts, whose one is the intrinsic efficiency and the other is the over veto ratio by the BVC and the CVC.
The intrinsic efficiency was estimated from the $K^- p \rightarrow K^0 n$ reaction using the hydrogen target data.
On the other hand, the over veto of the BVC and the CVC probably depend on the beam condition, so the ratio was estimated using the production data.
We identified the $d(K^-, n \pi^+ \pi^-)"n"$ final state,
and this final state was decomposed to the $k^0$ production, the $\Sigma_{forward}$ production, and the $d(K^-, n)"\pi^{\mp}\Sigma^{\pm}"$ state.
The $d(K^-, n)"\pi^{\mp}\Sigma^{\pm}"$ state includes two charge modes, which were decomposed at the bin-by-bin of the $d(K^-, n)"X"$ spectrum.
The decomposition of these modes was used to the templates that generated from the Monte Carlo simulation. 
The $d(K^-, n)"K^0 n"$ spectrum and the $d(K^-, n)"\pi^{\mp}\Sigma^{\pm}"$  spectra were corrected the acceptance of the CDS using the MC simulations
and were converted to the cross sections using the conversion factor which consists of the luminosity and the detector efficiencies. 

%% To do 内容に合わせて追記
%% To do conversion factorについて追記
