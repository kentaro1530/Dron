\chapter{Conclusion}
We measured $d(K^, N)"\pi\Sigma"$ reaction with $1GeV$/c $K^-$ beam at K1.8BR beamline of the hadron hoall in the J-PARC as the J-PARC E31 experiment.
We measured forward scattering scattering nucleon using the NC and PC.
Simultaneously,	decayed	particles are detected by the CDS surrounding the	liquid-$D_2$ target to identified final	state.
We identify $K^- d \rightarrow n \pi^+ \pi^- n$ final state and removed $K^0$ and forward-$\Sigma^{\pm}$ production, in which forward-$\Sigma^{\pm}$ means forward neutron decayed from $\Sigma^{\pm}$.
And, we obtain $d(K^-, n)"\pi^{\mp}\Sigma^{\pm}"$, which decomposed to $\pi^-\Sigma^+$ and $\pi^+\Sigma^-$ from missing mass of $d(K^-, n \pi^{\mp})"\Sigma^{\pm}"$.
We identify $\pi^-\Sigma^0$ final state from identify $d(K^-, p \pi^-)"\Sigma^0"$ and $d(K^-, p \pi^- \pi^-)"p"$.
At the result, We obtained \pimSp, \pipSm and \pimSz cross sections from the missing mass of the $d(K^, N)$ missing mass.

This reaction is considered as the 2-step reaction of $K^-N\rightarrow \bar{K}N$ scattering and $\bar{K}N \rightarrow \pi \Sigma$ scattering.
1-step reaction has large energy $\sim 2.05 GeV/$c and 2-step can allow to occur $\bar{K}N \rightarrow \pi \Sigma$  below the $\bar{K}N$ threshold.
Large energy 1-step reaction restricts the contamination from 1-step reaction in which a nucleon	emitted	as the spectator around the $\bar{K}N$ threshold.
Because the recoiled $\bar{K}$ has low momentum $\sim 0.25 GeV$/c around the $\bar{K}N$ threshold, the S-wave scattering is dominant in 2-step $\bar{K}N \rightarrow \pi \Sigma$ scattering,
which is confirmed from our data not to see obvious peak around the $\Sigma(1385)$ and $\Lambda(1520)$, which are P-wave and D-wave.
We can understand the reaction is the above the mechanism from the matching our data and theoretical calculations which adopts or covers around high energy $\bar{K}N$ scattering region around 1-step.

We decompose about the isospin $I=0$, $I=1$ and these interference term about 2-step scattering. %%ToDo スペクトルの説明
The so-called model.B is not matched spectra shape of $\pi^-\Sigma^+$ and $\pi^+ \Sigma^-$, especially below the $\bar{K}N$ threshold,
because this has large width of higher pole.
On the other hand, in model.B, obtained spectra are reproduced to change all $I=0$, $I=1$ and these interference term.
That means that $I=0$ component is important, but $I=1$ component is also important even though the component does not have pole in the region of interest.
That means that $I=0$ component is important, but $I=1$ component is also important even though the component does not have pole in the region of interest.
Above the threshold, $I=0$ $\pi^-\Sigma^0$ has large cross section and the interference term between $I=0$ and $I=1$ also appear as the difference of $\pi^-\Sigma^+$ and $\pi^+\Sigma^-$.
So, interference term is necessary to explain the these three spectra.

We obtain $\pi^-\Sigma^+$, $\pi^+\Sigma^-$ and $\pi^-\Sigma^0$ spectra via the $d(K^-, N)$ reaction,
which is considered 2-step reaction of $K^-N\rightarrow \bar{K}N$ and $\bar{K}N \rightarrow \pi\Sigma$.
These spectra provide all information to determine that $I=0$, $I=1$ and these interference term of $\bar{K}N \rightarrow \pi\Sigma$ scattering around the $\bar{K}N$ threshold.




% And we compare with the theoretical calculation of the dynamical-coupled framework, which can comprehensively treat 1-step and 2-step scattering.



