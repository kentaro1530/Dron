\section{Recent situation of $\Lambda(1405)$}
Wise et al. confirmed that the $\Lambda(1405)$ resonance pole is located below the $\bar{K}N$ threshold using a chiral Lagrangian with a strangeness $S=-1$ sector\cite{Wise}.
In the 2000's, the chiral unitary model proposed that $\Lambda(1405)$ is a dynamical generated molecular state consisting of two poles\cite{Jido}.
The higher pole is coupled to the $\bar{K}N$ channel, while the lower pole is coupled to the $\pi\sigma$ channel.
Therefore, they claim that the pole coupled to $\bar{K}N$ shifts to a higher position than the previously mentioned $1405$ MeV$/c$.
This approach is called the chiral unitary approach.

Photoproduction of $\Lambda(1405)$ was performed at LEPS in Spring-8 \cite{Niiyama}.
In that experiment, the $\gamma p \rightarrow K^+ \Lambda(1405)$ reaction at $E_{\gamma}=1.5-2.4$GeV was used to measure the $K^+$ scattered at $0.8<\cos\theta_{K^+}<1.0$.
They reported $\pi^+\Sigma^-$ and $\pi^-\Sigma^+$ and ovserved a difference between the two spectra.
This fact implies the existence of an interference term between $I=0$ and $I=1$, suggesting that $\Lambda(1405)$ is a dynamically generated state.

The CLAS Collaboration reported highly-statistical photoproduction of $\Lambda(1405)$.
The spectra of $\pi^-\Sigma^+$, $\pi^0\Sigma^0$, and $\pi^-\Sigma^+$ are measured
for total energies $2.55<W<2.85$GeV and $K^+$ scattering angles of $0.6<\cos\theta_{K^+}<0.9$ in the center-of-mass frame\cite{CLAS,CLAS2}.
T. Nakamura et al. reproduced those spectra finely well using the chiral unitary model, although the reaction mechanism is not simple \cite{Nakamura}.
L. Roca et al. constructed a model with two poles using photonuclear reactions with a potential with free parameters based on the chiral unitary model \cite{Roca}.
They deduced a high pole is located above $1405$ MeV$/c^2$, albeit with many parameters.

HADES Collaboration reported the spectra of $\Lambda(1405) \rightarrow \pi^{\mp}\Sigma^{\pm}$ using the $pp \rightarrow K^+ \Lambda(1405)$ reaction with a 1405GeV$/c$ $p$ beam \cite{HADES}.
The peak position of the spectra was located below $1400$GeV$/c$.
M. Hassanvand et al. obtained the mass and width of $\Lambda(1405)$ with $M=1405.1^{+11}_{-9}$MeV$/c^2$ and $\Gamma=62 \pm 10$MeV$/c^2$ by adapting phenomenological analysis \cite{HADES_pheno}.
On the other hand, J. Siebenson et al. argue that the data can be reproduced in a two pole structure such as the chiral unitary model \cite{HADES_ChU}.
