\begin{frame}{Reaction Identification}
  We expect these three reactions.
  \begin{itemize}
  \item $K^- d\rightarrow \pi^{\pm}"\Sigma^{\mp}" n_{forward}$ (Signal)
  \item $K^- d\rightarrow K^{0} n n$ (Quasi-elastic)
  \item $K^- d\rightarrow "n" \pi^{\pm} \Sigma^{\mp}_{forward}$ $\Sigma^{\mp}_{forward}\rightarrow n_{forward} \pi^{\mp}$
  \end{itemize}

  Identification method procedure.
  \begin{itemize}
  \item Three invariamt mass fitting of $\pi^+ \pi^-$, $n \pi^{\pm}$.\\
    $\rightarrow$ In this fitting, $K^- d\rightarrow \pi^{\pm}"\Sigma^{\mp}" n_{forward}$ was fixed.
    
  \item Missing mass of $d(K^-, n \pi^{\pm})"\Sigma^{\mp}"$ fitting.\\
    $\rightarrow$ In this fitting, $K^- d\rightarrow K^{0} n n$, $K^- d\rightarrow "n" \pi^{\pm} \Sigma^{\mp}_{forward}$ were fixed. \\
    $\rightarrow$ This fitting was performed bin-by-bin of $d(K^-, n)"X"$.\\
  \item These fitting was performed iterationary.
  \end{itemize}

  Fitting method was adopted below method.\\
  { \tiny
    \href{https://www.sciencedirect.com/science/article/pii/001046559390005W}{R. Barlow and C. Beeston, Comp. Phys. Comm. 77 (1993) 219-228 \\ \vspace{-3mm}
      "Fitting using finite Monte Carlo samples"}
  }
  
  { \tiny
    \href{https://www.sciencedirect.com/science/article/pii/S0010465508003652}{A.Nappi, Comp Phys. Comm. 180 (2009) 269-275 \\ \vspace{-3mm}
      "A pitfall in the use of extended likelihood for fitting fractions of pure samples in a mixed sample"}
  }
\end{frame}
