\begin{frame}{$K^0$ $\cos\theta_{K0}$ vs $mom_{K_0}$}
  \begin{tabular}{cc}
    \begin{minipage}{0.5\hsize}
      \begin{figure}
        Acceptance
        \begin{overpic}[ width=5cm ]{../pic/Run78/QE/K0_cos_mom_acc.eps}
          \linethickness{1pt}
          \put (25, 32){ \textcolor{red}{ \oval(20, 15) } }
        \end{overpic}
      \end{figure}
    \end{minipage}

    \begin{minipage}{0.5\hsize}
      \begin{figure}
        Data\\
        \begin{overpic}[ width=5cm ]{../pic/Run78/QE/K0_cos_mom_data.eps}
          \linethickness{1pt}
          \put (25, 32){ \textcolor{red}{ \oval(20, 15) } }
        \end{overpic}
      \end{figure}
    \end{minipage}
  \end{tabular}

  \vspace{ 10mm }
  \centering
  Data events concentrate around red frame($\cos\theta\sim-1,p=0.2GeV/c$). \\
  In this region, acceptance is smaller than central region of $\cos\theta$.
%  データの分布は赤枠($$\cos\theta\sim-1,p=0.2GeV/c$)に集中している。\\
%  この部分のアクセプタンスは$\cos\theta$の中央付近に比べて低い。
\end{frame}
