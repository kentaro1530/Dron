\begin{frame}{$K^- d \rightarrow n \pi^+ \pi^- "n"$ event fitting}
  \label{page:KNpipi_fit}
  
  \small
  \begin{enumerate}
  \item $K^- d \rightarrow K^0 "n" n_{detected}$ 
  \item $K^- d \rightarrow "n" \pi^{\mp} \Sigma^{\pm}_{forward}$  $\Sigma^{\pm}_{forward} \rightarrow \pi^{\pm} n_{detected}$ 
  \item $K^- d \rightarrow \pi^{\mp} "\Sigma^{\pm}" n_{detected}$ 
  \end{enumerate}
  
  \scriptsize  
  Each fittings was performed by data and template distribution reproduced by MC sim.\cite{TempFit}\\
  1. and 2. reactions was simulated using previous experimental angular distribution \cite{NPB90_349}\\
  These strength were estimated by fitting of invariant masses, $\pi^+ \pi^-$($K^0$)、$n \pi^+$($\Sigma^-$)、$n \pi^+$($\Sigma^+$) \\
  \vspace{3mm}
  $K^-d\rightarrow \pi^{\pm}"\Sigma^{\pm}"n_{detected}$ were separated by fitting of $d(K^-, n \pi^{\mp})"\Sigma^{\pm}"$ dist., which was performed bin-by-bin of $d(K^-, n)"X"$.\\
  \vspace{3mm}
  These fittings were performed iteration due to differenet event sample.\\
  In each fitting, the strength which was dicided by another fitting was fixed.
%  これらのフィッティングはイベントサンプルがちがうため別々に繰り返し行う。\\
%  それぞれのフィッティングで別のフィットで見積もられる強度は固定する。\\
  \begin{thebibliography}{99}
  \bibitem{TempFit} \href{https://www.sciencedirect.com/science/article/pii/001046559390005W}{R. Barlow and C. Beeston, Comp. Phys. Comm. 77 (1993) 219-228}
  \bibitem{NPB90_349} Nucl. Phys. B90, 349
  \end{thebibliography}
\end{frame}
