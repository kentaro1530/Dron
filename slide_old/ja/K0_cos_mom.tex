\begin{frame}{$K^0$ $\cos\theta_{K0}$ vs $mom_{K_0}$}
  \begin{tabular}{cc}
    \begin{minipage}{0.5\hsize}
      \begin{figure}
        アクセプタンス
        \begin{overpic}[ width=5cm ]{../pic/Run78/QE/K0_cos_mom_acc.eps}
          \linethickness{1pt}
          \put (25, 32){ \textcolor{red}{ \oval(20, 15) } }
        \end{overpic}
      \end{figure}
    \end{minipage}

    \begin{minipage}{0.5\hsize}
      \begin{figure}
        データ分布
        \begin{overpic}[ width=5cm ]{../pic/Run78/QE/K0_cos_mom_data.eps}
          \linethickness{1pt}
          \put (25, 32){ \textcolor{red}{ \oval(20, 15) } }

        \end{overpic}
      \end{figure}
    \end{minipage}
  \end{tabular}

  \vspace{ 10mm }
  \centering
  データの分布は赤枠($\cos\theta\sim-1,p=0.2GeV/c$)に集中している。\\
  この部分のアクセプタンスは$\cos\theta$の中央付近に比べて低い。
\end{frame}
