\begin{frame}{$K^0$解析方法}
  \small
  \begin{itemize}
  \item 予測される反応
    \begin{enumerate}
    \item $K^- d \rightarrow K^0 "n" n_{detected}$ (シグナル)
    \item $K^- d \rightarrow "n" \pi \Sigma_{forward}$  $\Sigma_{forward} \rightarrow \pi n_{detected}$ \\
    \item $K^- d \rightarrow \pi "\Sigma" n_{detected}$ 
    \end{enumerate}
  \item バックグランドの強度は前のフィッティングから見積もられている。\\
    $K^- d \rightarrow "n" \pi \Sigma_{forward}$は過去の実験の核分布が入っている。\\
    $K^- d \rightarrow \pi "\Sigma" n_{detected}$は実験結果に合うように見積もられている。\\ 
  \item 断面積への変換\\
    $K^0$の運動学($\cos\theta$ と運動量)でイベントごとに変換する
    \begin{eqnarray*}
      \frac{d^2\sigma}{d\Omega dM_{d(K^-, n)}}=F\sum M(\cos\theta, p)\times \frac{1}{A(\cos\theta, p)}
    \end{eqnarray*}
    アクセプタンスは$M_{"n"K^0}$の一様分布のモンテカルロから見積もる。 
  \end{itemize}  
\end{frame}
