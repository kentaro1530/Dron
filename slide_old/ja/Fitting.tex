\begin{frame}{$K^- d \rightarrow n \pi^+ \pi^- "n"$イベントフィッテイング}
  \label{page:KNpipi_fit}
  
  \small
  \begin{enumerate}
  \item $K^- d \rightarrow K^0 "n" n_{detected}$ 
  \item $K^- d \rightarrow "n" \pi^{\mp} \Sigma^{\pm}_{forward}$  $\Sigma^{\pm}_{forward} \rightarrow \pi^{\pm} n_{detected}$ 
  \item $K^- d \rightarrow \pi^{\mp} "\Sigma^{\pm}" n_{detected}$ 
  \end{enumerate}
  
  \scriptsize  
  それぞれの反応の強度はモンテカルロデータとのテンプレートフィッティングで行う\cite{TempFit}\\
  1.、2.の反応は過去の実験の$K^-p$反応の角分布をシュミレーションする\cite{NPB90_349}\\
  これらの強度は$\pi^+ \pi^-$($K^0$)、$n \pi^+$($\Sigma^-$)、$n \pi^+$($\Sigma^+$)の不変質量分布のフィッティングから見積もる。\\
  \vspace{3mm}
  $K^-d\rightarrow \pi^{\pm}"\Sigma^{\pm}"n_{detected}$は$d(K^-, n)"X"$のビン毎に
 $d(K^-, \pi^{\mp})"\Sigma^{\pm}"$の分布をフィッティングすることにより
  各荷電モードを分離して強度を見積もる。\\
  \vspace{3mm}
  これらのフィッティングはイベントサンプルがちがうため別々に繰り返し行う。\\
  それぞれのフィッティングで別のフィットで見積もられる強度は固定する。\\
  \begin{thebibliography}{99}
  \bibitem{TempFit} \href{https://www.sciencedirect.com/science/article/pii/001046559390005W}{R. Barlow and C. Beeston, Comp. Phys. Comm. 77 (1993) 219-228}
  \bibitem{NPB90_349} Nucl. Phys. B90, 349
  \end{thebibliography}
\end{frame}
