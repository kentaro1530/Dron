\begin{frame}{まとめ}
  \begin{itemize}
  \item バックグラウンドを考慮した$d(K^-, n)"n K^0"$の断面積を得た。
    \begin{itemize}
    \item アクセプタンスは$K^0$の角分布と運動量の2次元分布でイベントごとに補正する \\
    データは後方のアクセプタンスが低い場所に領域に分布しているため断面積ではシグナルが強調される\\
    \end{itemize}
  \end{itemize}
\end{frame}
