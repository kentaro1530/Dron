\begin{frame}{Contents \newline $d(K^-, n K^0)"n"$のバックグラウンドを \newline 考慮した断面積を得る}
  \label{content}
  \begin{itemize}
  \item 反応
    \begin{itemize}
    \item バックグラウンドの強度はテンプレートフィッティングで見積もる。
    \item シグナル : $K^- d \rightarrow K^0 "n" n_{detected}$
    \item バックグラウンド
      \begin{itemize}
      \item $K^- d \rightarrow \pi \Sigma_{forward} "n_{spec}"$ : $\Sigma_{forward} \rightarrow \pi n_{detected}$\\
        $K^-p$反応の過去の実験データーを使う \cite{NPB90_349}。
      \item $K^- d \rightarrow \pi "\Sigma" n_{detected}$\\
        これらのスペクトラムはデータを再現するよう$d(K^-, n)"X"$のビンごとのフィッティングの結果を使う
      \end{itemize}

    \end{itemize}
  \end{itemize}

  \begin{thebibliography}{99}
  \bibitem{NPB90_349} Nucl. Phys. B90, 349
  \end{thebibliography}
\end{frame}
