\begin{frame}{Contents \newline $d(K^-, n K^0)"n"$ analysis --- Background Subtraction}
  \begin{itemize}
  \item Signal\\
    $K^- d \rightarrow K^0 "n" n_{detected}$
  \item Background process.
    \begin{itemize}
    \item $K^- d \rightarrow \pi \Sigma_{forward} "n"$ : $\Sigma_{forward} \rightarrow \pi n_{detected}$
    \item $K^- d \rightarrow \pi "\Sigma" n_{detected}$
    \end{itemize}
  \item These strength was estimated by template fitting.
  \item Acceptance correction was performed event-by-event using \\
    2D histogram of $K^0$ $cos\theta$ and mom.\\
    \begin{equation*}
      \frac{d^2\sigma}{d\Omega dm_{d(K^-, n)"X"}}=N(m_{d(K^-, n)"X"})/A(\cos\theta_{K^0}, p_{K^0})
    \end{equation*}
    
%%  \item Demonstration os $K^0$ $cos\theta$ vs mom by MC sim.
  \item Updated $\frac{d^2\sigma}{d\Omega dm_{d(K^-, n)"X"}}$
  \end{itemize}

  %% Homeworks
  %% \begin{itemize}
  %% \item NC resolution $\sigma=150ps$ $\rightarrow$ $\sigma=170ps$.
  %% \item $K^0$ $cos\theta$ vs mom : $d(K^-, n)$ dependence.
  %% \item $d(K^-, n \pi^+ \pi^-)"n"$ tail : qualitative explanation.
  %% \end{itemize}
\end{frame}
