%\chapter{Discussion}

\if0
In the previous chapter, we obtained two results. One is the upper limits of the formation cross section of a strange dibaryon state in the deep bound region with more than 90 MeV binding energy. The other one is the intensity of the signal excess just below the $K^-+p+p$ threshold. These two should be simultaneously understood and compared with other theoretical and experimental studies.
\fi
\section{Comparison with theoretical calculations} \label{sec-compth}
The theoretically calculated spectra in the $^3$He($K^-,n$) reaction by Koike and Harada\cite{Koike:2009cx} are shown in Fig. \ref{fig-koike}. They applied various phenomenological potential to reproduce typical binding energies and widths of $K^-pp$ obtained by theoretical and experimental studies. The comparison of those spectra with our experimental one (Fig. \ref{fig-ncs}) is not straight forward since our spectra is somehow distorted by requesting at least 1 charged track in the CDS. 
As a consequence, the case of the Fig. \ref{fig-koike}(c),(d), which predict a peak or a cusp structure at the binding energy of $\sim$100 MeV with more than 1 mb/sr cross section, are clearly excluded. Considering the experimental resolution is as good as 10 MeV/$c^2$, we may have chance to observe a peak structure in the case of Fig. \ref{fig-koike}(b), but we did not. The spectrum in Fig. \ref{fig-koike}(a) seems most similar to our experimental spectrum qualitatively. Then, our experimental data favor a shallow binding with B.E. $<$40 MeV and moderately large width of $\Gamma$=50$\sim$100 MeV for the $K^-pp$ state in the frame work of Koike and Harada calculation. 

%Koike and Harada are now updating their calculation by employing chirally derived potential by Dote {\it et. al.}\cite{dote} and adding a contribution of spin 1 state, $\bar K^0 d$. The spin 1 state a  

\section{Comparison with other experiments} \label{sec-compex}
There are claims of the possible candidates of $K^-pp$ by FINUDA\cite{PhysRevLett.94.212303} and DISTO\cite{Yamazaki:2010ef}. The central values of the binding energies (B.E.) and the widths ($\Gamma$) are (B.E., $\Gamma$)=(115, 67) MeV and (105, 118) MeV, respectively. We did not observe any structure in those regions and upper limits of the formation cross section of the strange dibaryon state decaying into $\Lambda p$ were obtained as $\sim$0.2 mb/sr and $\sim$0.4 mb/sr, respectively. It contradict to the theoretical calculation by Koike and Harada\cite{Koike:2009cx}, where potentials which reproduce FINUDA and DISTO peak positions and widths make a distinct peak structure with more than 1 mb/sr integrated cross section in the $^3$He($K^-,n$) missing mass spectrum as shown in Fig. \ref{fig-koike}. 

One of the strength of our experiment is an almost background free condition in the bound region, while FINUDA and DISTO spectra would contain relatively large background from many complex processes. Our non-observation of a discrete peak structure is quite clear because we have few events there even with a minimum event selection.

The FINUDA spectrum \ref{fig-finuda} should be contaminated with the two-nucleon absorption process followed by $\Sigma-\Lambda$ conversion, $\Sigma^0\to\Lambda\gamma$ events and other final state interaction processes. In fact, they are now finalizing an analysis of the data obtained in the second data taking during 2006 and 2007, where such processes are considered in the global fit of the $\Lambda p$ distributions\cite{Agnello:2013bn}. % Although they still have a small excess at around 2300 MeV/$c$ in the $\Lambda p$ invariant mass distribution from $^6Li$, the majority of the $\Lambda p$ events are explained without introducing $K^-pp$. Therefore further understanding of the background processes is of primary importance.

In the DISTO case, their spectrum in Fig. \ref{fig-disto} was obtained by the deviation to the uniform phase space distribution of the $\Lambda p K^+$ final state. Contributions from $N^*$ resonances in the $pp$ collision was pointed out by COSY-TOF\cite{AbdElSamad:2010bf}, where the final state products are also $\Lambda p K^+$, via $pp\to pN^*, N^*\to K^+\Lambda$. These $N^*$ resonances should make deviation from the uniform distribution in the phase space and modify DISTO spectrum to some extent. 

Therefore for both FINUDA and DISTO cases, the background processes should be carefully studied further.
\\
