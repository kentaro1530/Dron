\chapter{Introduction}
\section{Kaonic nuclear bound state}
Meson-nuclear bound systems have unique importance for the study of strong interactions and nuclear medium effects. Exotic atoms, in which a negative hadron is bound by the Coulomb interaction, have been studied intensively by means of x-ray spectroscopy for this purpose. However, they are only sensitive to the interaction at the peripheral region of the nucleus. Much more direct information can be obtained by planting mesons deep inside the nucleus.

For pions, a breakthrough experiment had been done at GSI, where deeply bound atomic state was directly populated via the (d,$^3$He) reaction and observed as a distinct peak\cite{Yamazaki:1998kh,Gilg:2000hk,Itahashi:2000ks,Suzuki:2004dr}.% The observed peak position indicates the $\sim$20 MeV increase of the effective pion mass at the center of the $^{207}$Pb nucleus. 
The deduced reduction of the isovector parameter of the $s$-wave pion-nucleon potential at the normal nuclear density was considered to be an evidence for the partial restoration of chiral symmetry.

In the anti-kaon sector, there are theoretical calculations to predict the reduction of the effective $K^-$ mass in a nuclear medium\cite{Waas:1997je,Waas:1996hy}. As a possible candidate to investigate such in-medium effect, recent theoretical studies predict the existence of kaonic nuclear bound state as a consequence of the strongly attractive $\bar K N$ interaction in the isospin $I$=0 channel. However, the existence of such bound state has not been established and the situation in turn suggests a lack of our understanding of the $\bar K N$ interaction, especially in the sub-threshold region.

\subsection{$\bar KN$ interaction}
The $\bar K N$ interaction has been investigated by scattering experiments\cite{Martin:1980qe} and x-ray measurements of anti-kaonic hydrogen. These two methods had long-standing inconsistency until late 1990's. An improved x-ray measurement at KEK resolved this inconsistency and established the attractive $\bar K N$ interaction in $I$=0\cite{PhysRevLett.78.3067}. The x-ray measurement at KEK was later confirmed by the DEAR experiment\cite{PhysRevLett.94.212302} and the SIDDHARTA experiment\cite{Bazzi:2011wva} at DA$\Phi$NE in Italy.  

But the $\bar KN$ interaction, especially at the sub-threshold region, is still not understood well because of the existence of $\Lambda(1405)$ resonance. $\Lambda(1405)$ is a baryon species with strangeness $S$ = −1, spin-parity $J^P = (1/2)^-$, and $I$ = 0. While its assignment to an ordinary 3-quark state is difficult, it has been interpreted as a quasi-bound state of $K^-p$ with a binding energy of 27 MeV\cite{PhysRevC.65.044005}. On the other hand, a recent calculation in the frame work of a chiral unitary model claims that $\Lambda(1405)$ is dynamically generated by the $\Sigma\pi-\bar KN$ coupling and consists of two poles couples to the $\pi\Sigma$ state and the $\bar KN$ state\cite{hyodo:035204}. As a consequence, the resonance position of the $\bar KN\to\pi\Sigma$ channel sits at about 1420 MeV and the binding energy is as shallow as 15 MeV. The difference in the interpretation of $\Lambda(1405)$ of course largely affects on the calculation of $\bar K$-nucleus systems.
Experimental study of the structure of $\Lambda(1405)$ is now being progressed extensively. New data of $\pi\Sigma$ line shapes are available in different charge channels produced by $\gamma$-induced reaction with CLAS\cite{Moriya:2013eb,Moriya:2013hq} and LEPS\cite{niiyama:035202} spectrometers, and $pp$ collisions at COSY\cite{Zychor2008167} and with HADES spectrometer at GSI\cite{Agakishiev:2013ba}. These data together with a coming kaon-induced data from J-PARC\cite{Noumi:872U_z1G} would help us to understand the nature of $\Lambda(1405)$.

\subsection{$\bar K$-nucleus interaction}
$\bar K$-nucleus interaction has been studied by the systematic measurement of kaonic atom x-rays. Intensive measurements of kaonic atom x-rays were performed in 1970's and 1980's. A global fit of those data with a density-dependent potential resulted in a deep potential in the real part -160$\sim$-200 MeV\cite{Batty1997385}. On the other hand, chirally motivated calculations also reproduced the x-ray data with a relatively shallow potential -40$\sim$-60 MeV\cite{PhysRevC.61.055205}. The large discrepancy in the potential depth is due to the fact that the atomic x-ray data probe only surface of the nucleus and indicates that we do not understand the $\bar K$-nucleus interaction qualitatively. A much more high precision x-ray data, however, may provide new insight for quantitative understanding of the $\bar K$-nucleus interaction.

\subsection{Prediction of kaonic nuclear bound state}
The idea of kaon nuclear bound state was proposed by Nogami in 1963 for the first time\cite{Nogami:1963dw} soon after the discovery of $\Lambda(1405)$\cite{Dalitz:1959dn}. A hypothesis of the possible existence of deeply-bound $\bar K$ states was advocated by Wycech in 1986 based on the kaonic atom data\cite{Wycech:1986tx}. However, we should wait for the establishment of the attractive $\bar KN$ interaction by the new kaonic hydrogen x-ray data\cite{PhysRevLett.78.3067} to motivate the further study on this topic. The recent extensive studies on the kaonic nuclei were triggered by Akaishi and Yamazaki, who carried out a pioneering work on the light kaonic nuclear systems\cite{PhysRevC.65.044005}\cite{Yamazaki:2002wp}. Their calculation was based on a phenomenological $\bar KN$ interaction, where $\Lambda(1405)$ was assumed to be a $K^-p$ bound state with $I$=0. As a consequence, they predicted a narrow ($\sim$20 MeV) state below the $\Sigma\pi$ emission threshold in the case of $K^-$ in $^3$He nuclei. They also pointed out quite unique feature of the kaonic nuclear system that the central density of such system becomes 5-10 times higher than the normal nuclear density\cite{Yamazaki:2004gq}. 

\section{Experimental situation}
Many experiments to search for the kaonic unclear bound states were performed in the past decade and the effort still continues to reveal the nature of kaonic nuclear states. Here, a part of the experimental searches are reviewed.
\subsection{Stopped $K^-$ experiment at KEK}
In early stage, four-body bound systems ($K^-$ in $^3$He) were searched for by a series of experiment with the $^4$He($K^-_{stopped},N$)X reactions at KEK ( KEK PS E471\cite{Suzuki:2004tu},E549\cite{Sato2008107}\cite{Yim:2010ed}). They excluded the formation branch as large as 1\% per stopped kaon for narrow strange tribaryon states. The inclusive proton spectrum is shown in Fig. \ref{fig-e549sato}. Although the statistics and the experimental resolution were excellent, they suffered from rather large background contribution in the region of interest. Their origins were attributed to the hyperon decays or multi-nucleon absorptions. An exclusive analysis based on these data shows that $^4$He was still too complicated to obtain the conclusive evidence of the formation of kaonic nuclear states\cite{Suzuki:2008bc}
\begin{figure}[]
\begin{center}
\includegraphics[width=10cm]{./illustrator/e549sato.eps}
\caption[The proton inclusive missing mass spectrum obtained in the KEK E549/570 experiment.]{The proton inclusive missing mass spectrum obtained in the KEK E549/570 experiment\cite{Sato2008107}.}
\label{fig-e549sato}
\end{center}
\end{figure}
 
\subsection{In-flight $K^-$ experiment}
The ($K^-_{\rm in-flight}, N$) method to populated kaonic nuclear bound states was proposed by Kishimoto\cite{Kishimoto:1999ga}. He also applied it to experiments at BNL\cite{Kishimoto2005383} and at KEK\cite{Kishimoto:2007kr}. The spectrum of the $^{12}$C($K^-,N$) reactions in the KEK E548 experiment are shown in Fig. \ref{fig-e548kishimoto}, which indicated the strongly attractive  $\bar K$-nuclear interaction. The depth of potential around -190 MeV and -160 MeV was obtained for the $^{12}$C($K^-,n$) and the $^{12}$C($K^-,p$) reactions, respectively, as a result of the fitting with a spectrum calculated by the Green's function method. However, there is a claim that the spectrum shape can be reproduced with a moderate potential depth of -60 MeV\cite{Ramos:2010ut}.

It should be noted that spectra in Fig. \ref{fig-e548kishimoto} have few events in the deep bound regions, which indicates that non-mesonic two-nucleon absorption processes are substantially suppressed in in-flight $K^-$ reaction.
\begin{figure}[]
\begin{center}
\includegraphics[width=7cm]{./illustrator/e548kishimoto.eps}
\caption[Missing mass spectra of the $^{12}$C($K^-,N$) reactions in the KEK E548 experiment]{Missing mass spectra of the $^{12}$C($K^-,n$) reaction (upper) and $^{12}$C($K^-,p$) reaction (lower) in the KEK E548 experiment. The solid curves represent the calculated best fit spectra for potentials with Re(V)=-190 MeV and Im(V)=-40 MeV (upper) and Re(V)=-160 MeV Im(V)=-50 MeV (lower). The dotted curves represent the calculated spectra for Re(V)=-60 MeV and Im(V)=-60 MeV. The dot-dashed curves represent a background process. Taken from Ref. \cite{Kishimoto:2007kr}.} 
\label{fig-e548kishimoto}
\end{center}
\end{figure}  

\subsection{Claims of strange dibaryon candidates}
The main concern was soon moved to the strange dibaryon system. The lightest and the most fundamental kaonic nucleus is expected to be so-called $K^-pp$ state. It is more generally expressed as [$\bar K \otimes\{NN\}_{I=1}]_{I=1/2}$, where the maximum combinations of the strongly attractive $I$=0 channel is realized among the three-body systems. In this thesis, such dibaryon state with strangeness $S=-1$ is denoted as $K^-pp$.

\subsubsection{FINUDA experiment}
The FINUDA collaboration at DA$\Phi$NE reported a possible evidence of a deeply-bound $K^-pp$ state in the invariant mass distribution of back-to-back $\Lambda p$ pairs from the stopped $K^-$ absorption on $^6$Li, $^7$Li and $^{12}$C\cite{PhysRevLett.94.212303}. The observed bump structure is shown in Fig. \ref{fig-finuda}. The spectral fitting gives the binding energy of $115^{+6}_{-6}$(stat.)$^{+3}_{-4}$(syst.) MeV and the width of $67^{+14}_{-11}$(stat.)$^{+2}_{-3}$(syst.) MeV. However, there is a criticism that the bump structure comes from quasi-free two-nucleon reaction followed by secondary processes\cite{Magas:2006kh,Pandejee:2010wr}. In fact, they are now finalizing an analysis of the data obtained in the second data taking during 2006 and 2007, where such processes are considered in the global fit of the $\Lambda p$ distributions\cite{Agnello:2013bn}. Another criticism sometimes quoted to the result is that the acceptance of the given spectrum is too narrow for the clear determination of the background below the suggested peak structure.

\begin{figure}[]
\begin{center}
\includegraphics[width=10cm]{./illustrator/finuda.eps}
\caption[$\Lambda p$ invariant-mass spectrum in the FINUDA experiment]{Invariant mass of a $\Lambda$ and a proton in back-to-back correlation ($\cos\theta^{Lab}<-0.8$) from light targets before the acceptance correction in the FINUDA experiment. The inset shows the result after the acceptance correction. Taken from Ref.\cite{PhysRevLett.94.212303}}
\label{fig-finuda}
\end{center}
\end{figure}

\subsubsection{DISTO experiment}
A large formation probability of the $K^-pp$ bound state in $pp$ collision was predicted by Yamazaki, {\it et al.}\cite{Yamazaki:2007vr}. They calculated the process $\Lambda^*$and $p$ form to $K^-pp$ following the elementary process, $p+p\to p+\Lambda^*+K^+$. They insisted the $\Lambda^*p$ sticking become dominant due to the matching of the small impact parameter (large momentum transfer) with the compact bound state. 

The DISTO collaboration at SATURNE recently re-analyzed their data on the exclusive $p+p\to p+\Lambda+K^+$ events at $T_p$= 2.85 GeV\cite{Yamazaki:2010ef}, assuming $K^-pp$ decaying into $\Lambda p$. The $K^+$ missing mass spectrum obtained as the deviation to the uniform phase space distribution of the $\Lambda p K^+$ final state is shown in Fig. \ref{fig-disto}(left). A broad peak structure exhibits there at a mass of 2265.2 MeV/$c^2$ and a width of 118.8 MeV/$c^2$ and they associated it to the dibaryon state with strangeness -1 as X(2265). They also analyzed the data at $T_p$= 2.5 GeV and found no corresponding peak ($X(2265)$) observed at $T_p$= 2.85 GeV\cite{Kienle:2012dm}. 

If $X(2265)$ follows the excitation function in a semi-empirical universal form of Sibirtsev\cite{Sibirtsev:1995hv}, the peak should be observed also at $T_p$=2.5 GeV as shown in Fig. \ref{fig-disto}(right). In addition, they ignored $N^*$ resonances, whose contributions in the $pp$ collision were pointed out by the COSY-TOF experiment\cite{AbdElSamad:2010bf}. In the $N*$ production, the final state products are $\Lambda p K^+$, via $pp\to pN^*, N^*\to K^+\Lambda$ as is the case of $K^-pp$ production. These $N^*$ resonances should make deviation from the uniform distribution in the phase space and modify the DISTO spectrum to some extent. 

\begin{figure}[]
\begin{minipage}{0.5\textwidth}
\begin{center}
\includegraphics[width=\columnwidth]{./illustrator/disto.eps}
\end{center}
\end{minipage}
\begin{minipage}{0.5\textwidth}
\begin{center}
\includegraphics[width=\columnwidth]{./illustrator/disto2.eps}
\end{center}
\end{minipage}
\caption[$K^+$ missing mass spectrum observed in DISTO and relative excitation functions of the related reactions.]{(left)$K^+$ missing mass spectrum of $p+p\to\Lambda+K^++p$ channel observed in DISTO\cite{Yamazaki:2010ef}. Large momentum transfer protons and kaons are selected. (right) Relative excitation functions in arbitrary units of the reactions $p+p\to p+\Lambda+K^+, \to p+\Sigma^0+K^+, \to X(2265)+K^+, \to p+\Sigma^{0*} +K^+$ and $\to p+\Lambda^* +K^+$. The observed relative cross-sections for X(2265) at 2.50 and 2.85 GeV are shown by large red circles, and the expected one at 2.50 GeV relative to that at 2.85 GeV is shown by a green star. Taken from Ref. \cite{Kienle:2012dm}.}
\label{fig-disto}
\end{figure}

\subsubsection{OBELIX experiment}
The OBELIX experiment, which studied antiproton annihilation on $^4$He at rest, reported narrow peak structure in the $p\pi^-p$ and $p\pi^- d$ invariant mass spectra with a huge combinatorial backgound\cite{Bendiscioli:2007tn}. They insisted the former peak is the $K^-pp$ bound state with the -160.0$\pm$4.9 MeV binding and the $<$24.4$\pm$8.0 MeV width. Since their poor experimental resolution cannot identify $\Lambda$ and $K^0_s$, the claim is not conclusive.

%\subsubsection{FOPI experiment}


\subsection{Other searches for the $K^-pp$ bound state}
\subsubsection{$\gamma$-induced production at LEPS}
The $\gamma$-induced production off deuteron, $\gamma d\to K^+\pi^-X$ reaction at $E_\gamma=$1.5-2.4 GeV is analyzed with the LEPS/Spring-8 data\cite{Tokiyasu:2013mwa}. They found no significant bump structure in the region from 2.22 to 2.36 GeV/$c^2$, and the upper limits of the differential cross section for the $K^-pp$ bound state production were determined to be 0.1-0.7 $\mu$b with 95\% confidence level.
\subsubsection{HADES experiment}
The HADES experiment at GSI searched for $K^-pp$ in the same reaction as the DISTO experiment but with a higher kinetic energy of the proton beam, $T_p$=3.5 GeV. Although their analysis has not been finalized, they have not observed the DISTO-like peak so far\cite{Fabbietti:2013ux}.

\subsubsection{$d(\pi^+, K^+$) reaction at J-PARC E27}
The J-PARC E27 collaboration searched for $K^-pp$ in the $d(\pi^+,K^+)$ reaction\cite{Nagae:sGd-9z2X}. In this reaction, $\Lambda(1405)$ is produced on a neutron in a deuteron and the $\Lambda^*$ is considered to merges with a proton in the deuteron to form $K^-pp$\cite{Yamazaki:2007vr}. They constructed a range array counter to tag fast protons from $K^-pp$ decay, while the scattered $K^+$ were analyzed with the SKS spectrometer. The analysis is still ongoing. 
\if0
\subsubsection{AMADEUS experiment}
\fi
\section{Theoretical situation on $K^- pp$}
Many few-body calculations of the $K^-pp$ system have been progressed. Their results for binding energies and widths of the $K^-pp$ state are summarized in Table \ref{tab-kppsummary}. All calculations agree on the existence of the $K^-pp$ bound state. However the binding energy and the width are quite diverge.

The key issue in these theoretical calculations is the extrapolation of the $\bar KN$ interaction into the region far below threshold. As already described, chiral SU(3) based calculations dynamically generate the $\Lambda(1405)$ as a consequence of coupled-channel dynamics and the resonance position couples to $\bar KN$ shifted to about 1420 MeV, while $\Lambda(1405)$ is directly interpreted as a $\bar KN$ bound state in the energy-independent potentials. The former case results in substantially weaker $\bar KN$ interaction in $I$=0 compared to the later case. Therefore, the obtained binding energies of the $K^-pp$ system are smaller in the chiral, energy dependent calculations than those in non-chiral, static calculations. 

The calculation methods made less difference in the binding energy and the width. In the variational calculations (var.), the complex $\bar KN$ interaction accounted for the $\bar KN$-$\pi\Sigma$ two-body coupled channels, while three-body Faddeev calculations were performed in the $\bar KNN$-$\pi YN$ system.

None of the few-body calculations predicted the binding energy over 100 MeV despite the experimental observations of DISTO and FINUDA. Furthermore, the main decay channel of $\bar KNN\to \pi\Sigma N$ close at the binding energy around 100 MeV, and thus a deeply binding state is naturally expected to have narrow width as the originally-predicted $K^-ppn$ bound state by Akaishi and Yamazaki. Note that only mesonic decay amplitude, $\bar K NN\to \pi YN$, is taken into account in Table \ref{tab-kppsummary}. %The two-nucleon absorption process $\bar KNN\to YN$ would add, conservatively, 20 MeV to the overall width\cite{}.

\if0
The measure difference among the theoretical calculations comes from the treatment of the two-body interactions. 
Also in the theoretical side, many few-body calculations of $K^-pp$ system have been progressed. Yamazaki and Akaishi applied a variational ATMS method with a potential constructed assuming $\Lambda(1405)$ to be $K^-p$ quasi bound state and obtained the binding energy ($B$) of 48 MeV and the decay width($\Gamma$) of 61 MeV\cite{Yamazaki:2002wp}. They obtained the average distance between the nucleons of 1.90 fm, which is much smaller than that in a deuteron 3.90 fm). 

Shevchenko {\it et al.} performed a coupled-channel Faddeev calculation of an $I=1/2, J^\pi=0^- \bar KNN$ quasi bound state in the $\bar KNN-\pi\Sigma N$ system\cite{Shevchenko:2007ie,Shevchenko:2007tm}. They obtained $B$=50-70 MeaV and $\Gamma$=90-110 MeV, depending on the scattering length of $K^-p$ interaction. They also assumed $\Lambda(1405)$ to be $K^-p$ quasi bound state

Similarly, Ikeda et al. obtained the energy of the $\bar KNN$ resonance by the $\bar KNN-\pi YN$ coupled-channel Faddeev calculation.

and they failed to explain the experimental observation of DISTO and FINUDA.
\fi
\begin{table}[]
\caption[Calculated $K^-pp$ binding energies $B$ \& widths $\Gamma$.]{Calculated $K^-pp$ binding energies $B$ \& widths $\Gamma$ (in MeV). Taken from Ref. \cite{Gal:2013wj}}
\begin{center}
\begin{tabular}{lccc|cccc} 
\hline
& \multicolumn{3}{c}{chiral, energy dependent} &\multicolumn{4}{c}{non-chiral, static calculations} \\
& var. \cite{Barnea:2012gk} & var. \cite{Dote:2009um}& Fad. \cite{Ikeda:2010bd}& var. \cite{Yamazaki:2002wp}& Fad. \cite{Shevchenko:2007ie,Shevchenko:2007tm}& Fad. \cite{Ikeda:2009kv,Ikeda:2007vh}& var. \cite{Wycech:2009tg}\\
\hline
$B$ & 16 & 17-23 & 9-16 & 48 & 50-70 & 60-95 & 40-80 \\
$\Gamma$ & 41 & 40-70 & 34-46 & 61 & 90-110 & 45-80 & 40-85 \\
\hline
\end{tabular}
\end{center}
\label{tab-kppsummary}
\end{table}%

\section{Our approach}
As is reviewed in the previous sections, neither the existence nor the nature of $K^-pp$ bound state has been established. Although there are experimental claims of the possible $K^-pp$ candidates, the situation is far from to be settled since the interpretation of the disagreement of the observed peak positions and widths, and the deep binding with rather large width is not well explained by the theoretical calculations so far. In addition, different interpretations other than the $K^-pp$ assumption are available for the FINUDA observation. For the DISTO case, the absence of the peak in different incident kinetic energies including the HADES experiment is not understood. The $N^*$ contributions should be also understood. 

The previous experiments also told us that an exclusive measurement is essential for the clear identification of the observed structure. In addition, an in-flight reaction much suppresses or kinematically separates the background processes such as multi-nucleon absorptions and hyperon decays in the missing-mass measurement. 

These ideas are realized in the present experiment, J-PARC E15\cite{Iwasaki:yGk5b3Xx}. We employ in-flight ($K^-,n$) reaction at 1 GeV/c~, where the cross-section of the elementary $K^-N$ reaction is a maximum as shown in Fig. \ref{fig-knxsection}. With this beam energy, a forward going neutron have 1.2$\sim$1.4 GeV/c momentum, while the $K^-pp$ cluster goes backward with 0.2$\sim$0.4 GeV/c momentum according to its binding. The in-flight kaon reaction is still missing one among the $K^-pp$ searches but one of the most important reactions. Another point is a liquid $^3$He target employed to have more chance to detect all the particles so that we can reconstruct full kinematics of the reaction. The $K^-pp$ bound state is investigated via invariant-mass spectroscopy of the expected decay, $K^-pp \to \Lambda p \to \pi^-pp$, in addition to the formation via missing-mass spectroscopy by using the emitted neutron. An intense kaon beam now only available at J-PARC makes such a measurement possible. The details of the experimental method are described in the next chapter.

\begin{figure}[]
\begin{center}
\includegraphics[width=7cm]{./illustrator/knxsection.eps}
\caption[Total cross section of the elementally $K^-N$ reactions.]{Total cross section of the elementally $K^-N$ reactions. Taken from Particle Data Group.}
\label{fig-knxsection}
\end{center}
\end{figure}  

\subsection{Theoretical spectra of the $^3$He($K^-, n$) reaction}
For the present experiment, two theoretical spectra are available. One is done by Koike and Harada\cite{Koike:2009cx}. Their calculation resulted in the enhancement of the $K^-pp$ formation cross section at forward neutron angle as shown in Fig. \ref{fig-koike}. 
%This is because of the backward-recoiling nature of the kaon beam in the present reaction. 
In their calculation, phenomenological optical potentials were used with a form of
\begin{eqnarray}
U^{opt}(E;{\mathbf r}) = [V_0 +iW_0f(E)]\exp[-({\mathbf r}/b)^2],
\end{eqnarray}
where $V_0$ and $W_0$ are adjusted parameters to reproduce the results for the the binding energy and width of the $K^-pp$ state in theoretical predictions or experimental data as listed in Table \ref{tab-koike}. $f(E)$ is a phase space suppression factor 
introduced by Mares {\it et al.}, which accounts for the energy dependence here, and $b$ is a range parameter fixed to $b=1.09$ fm.  With a potential which reproduces the FINUDA observation ($D_2$ in Table \ref{tab-koike}), we expect clear peak structure in the $^3$He($K^-,n$) missing mass spectrum as shown in Fig. \ref{fig-koike}(d). The integrated cross section in the peak region is more than 1 mb/sr, which is easy to be observed experimentally. 

The other theoretical spectrum was derived by Yamagata-Sekihara {\it et al.}\cite{YamagataSekihara:2009bw} in the frame work of chiral unitary model. Their spectrum in Fig. \ref{fig-yamagata} shows a small bump structure as a result of weak $\bar KN$ interaction in a chiral unitary model, which may be difficult to identify in the present experiment. Another interesting result from their calculation is that the structure in the bound region may become much clear if we can tag a $\Sigma$ in the expected decay of $K^-pp\to \pi\Sigma p$.
\begin{figure}[]
\begin{center}
\includegraphics[width=12cm]{./illustrator/koike.eps}

\caption[Calculated inclusive spectra for the $^3$He($K^-_{\rm{in-flight}},n$) reaction at $p_{K^-}$=1.0 GeV/$c$ and $\theta_{lab}=0^\circ$ by Koike and Harada.]{Calculated inclusive spectra for the $^3$He($K^-_{\rm{in-flight}},n$) reaction at $p_{K^-}$=1.0 GeV/$c$ and $\theta_{lab}=0^\circ$ as a function of energy $E$ of the $K^-pp$ system measured from the $K^- + p + p$ threshold for
potentials (a) A, (b) B, (c) C, and (d) D2. The potential parameters are summarized in Table \ref{tab-koike}. Solid and dashed curves denote the inclusive spectra for the energy-dependent $U^{opt}(E)$ and energy-independent $U^{opt}_0$ potentials, respectively. The dotted curve denotes the L = 0 component in the inclusive spectrum for $U^{opt}(E)$. The vertical line at E = 0 MeV indicates the $K^- + p + p$ threshold. Taken from Ref. \cite{Koike:2009cx}.}
\label{fig-koike}
\end{center}
\end{figure}  

\begin{table}[]
\caption[Potential parameters used in the calculation by Koike and Harada.]{Potential parameters used in the calculation by Koike and Harada\cite{Koike:2009cx}. Branching rates of the one-nucleon $K^-$ absorption process are taken to be $B^{(\pi\Sigma N)}$ =0.7 and $B^{(\pi\lambda N)}$=0.1, respectively, and the branching rate of the two-nucleon $K^-$ absorption process is $B_2^{(YN)}$ = 0.2. Values in parentheses are for the imaginary parts of the energy-independent potentials. All values in MeV.}
\begin{center}
\begin{tabular}{l|ccccccl} 
\hline\hline
potential	&	$V_0$	&	$W_0$	&	\multicolumn{2}{c}{Without $B_2^{(YN)}$}			&	\multicolumn{2}{c}{With $B_2^{(YN)}$}			&	Ref.\\
	&		&		&	B.E.	&	$\Gamma$	&	B.E.	&	$\Gamma$	&	\\
\hline														
$A$	&	-237	&	-128(-120)	&	21	&	70	&	15	&	92	&	DHW\cite{Dote:2009um}\\
$B$	&	-292	&	-107(-86)	&	48	&	61	&	45	&	82	&	YA\cite{Yamazaki:2002wp}\\
$C$	&	-344	&	-203(-147)	&	70	&	110	&	59	&	164	&	SGM\cite{Shevchenko:2007ie,Shevchenko:2007tm}\\
$D_2$	&	-404	&	-213(-47)	&	118	&	19	&	115	&	67	&	FINUDA\cite{PhysRevLett.94.212303}\\
\hline\hline									
\end{tabular}
\end{center}
\label{tab-koike}
\end{table}%

\begin{figure}[]
\begin{center}
\includegraphics[width=0.9\columnwidth]{./illustrator/yamagata.eps}
\caption[Calculated results of $^3$He($K^-,n$) reaction spectra by Yamagata-Sekihara {\it et al.}.]{Calculated results of $^3$He($K^-,n$) reaction spectra at
$T_{K^-}$ = 600 MeV ($P_{K^-}$ = 976 MeV/$c$) including both $K^-pp$ and
$\bar K^0pn$ in the final states are shown in (a) total spectra and (b) conversion part at $\theta_n^{lab}=0^\circ$ for the $s$-wave chiral unitary optical potential. Dashed and dotted lines indicate the contributions from $K^-pp$ formation and $\bar K^0 pn$ formation, respectively. Solid lines are the sum of the both contributions. Take from Ref. \cite{YamagataSekihara:2009bw}.}
\label{fig-yamagata}
\end{center}
\end{figure}  

\subsection{First-stage physics run}
The calculation by Koike and Harada indicates our experiment can provide a crucial test for the experimental observations by FINUDA and DISTO. If the observed structures are truly attributed to the formation of $K^-pp$, we can observe a clear peak in the neutron missing-mass spectrum even with a moderate kaon beam intensity. It takes advantage of the relatively large formation cross section and the suppressed background with the in-flight reaction kinematics. Thus we decided to take a staging strategy and focus on the forward neutron analysis in a first stage physics run to confirm or refute the potentials corresponding to FINUDA and DISTO observations in the framework of the calculation by Koike and Harada. 

The requested beam time for the first stage physics run was only 2\% of the original proposal. Unfortunately, the executed beam time was further halved because of the facility condition.

\section{Thesis overview}
The present thesis is dedicated to an analysis of a neutron spectrum in the $^3$He($K^-,n$) reaction with the data obtained in the first stage physics run of the J-PARC E15 experiment. 
Firstly, experimental details such as the apparatus, the experimental conditions and data summary are described in Chapter 2. Then, a procedure of the detector analysis is presented and the performance of the measurement is evaluated in Chapter 3. The obtained neutron missing mass spectrum is presented and discussed in Chapter 4, focusing on the $K^-pp$ binding region. Finally, a conclusion is given in Chapter 5.
\\

The author took a major role throughout the present experiment; construction of a new spectrometer system, commissioning of a kaon beam line and detectors, physics data taking and the data analysis stage. The major contributions are as follows:

\begin{itemize}
\item Development of the $^3$He cryogenic target
\item Construction of the forward time-of-flight counters
\item Proposal, development, and installation of the beam defining counter
\item Construction of the trigger logic circuit
\item Development and maintenance of the data acquisition system
\item Planning and execution of the beam-line tuning and the physics-data taking
\item Development of all the analysis method including the programming. 
\item All the parameter optimizations in the analysis.
\item Implementation of the realistic geometry in the Geant4-based Monte Carlo simulation program.
\end{itemize}


