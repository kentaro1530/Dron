%\chapter{Detector analysis}
\section{Overview}
This chapter is dedicated for the data analysis to obtain the missing mass spectra in the $^3$He($K^-, n$) reaction. 

The first part of this chapter describes analyses of individual detectors. First, common procedures for all the counters are described in Sec. 3.2. Then, an analysis of the kaon beam is presented and the selection of the kaon beam is defined in Sec. 3.3. The analysis of the cylindrical detector system (CDS) comes to the next. A tracking method of the cylindrical drift chamber (CDC) is presented to reconstruct a reaction vertex, and a particle identification is performed to reconstruct $K^0_s$ and $\Lambda$ peaks for the performance check in Sec. 3.4. An analysis of the forward neutron appears in Sec. 3.5. 

In the later part of this chapter, an analysis with a whole system is presented. The resolution and the precision of scale in the $^3$He($K^-,n$) missing mass are evaluated in Sec.~3.6 and Sec.~3.7, respectively. Finally in Sec. 3.8, the normalization factors of the missing mass spectrum is evaluated including the neutron detection efficiency with the NC.

\subsection{Definition of the coordinates}
In the present analysis, we employ the right-handed system, where $z$-axis is defined by the (designed) beam-axis, and then $x-$ and $y$-axis represent horizontal and vertical positions, respectively. A positive $x$ position corresponds to right-hand side from the beam axis when we look from downstream of the beam line.
\section{Common procedures to all detectors}
\subsection{TDC data conversion to the time}
To convert a TDC channel to time, conversion parameters, i.e., the time gain, were obtained channel by channel with a time calibrator (ORTEC 462). The calibration data were recorded before the experiment. We used linear functions for the wire chambers and quadratic curves for the scintillation counters to describe conversion parameters of their TDCs.% A measured integral non-linearity of a TDC channel for a segment of the neutron counter is shown in Fig.\ref{}, together with the time difference between the measurement and estimation with a quadratic conversion curve.

\subsection{ADC data conversion to the energy}
For the conversion from an ADC channel to energy, 1 GeV/$c$ $\pi^-$ beams, whose energy deposit on a scintillator is $\sim 2$ MeV/1 cm, is used. To calibrate ADCs of the forward counters, we took a dedicated run to irradiate the pion beam to all the segments by scanning the Ushiwaka field before the experiment and once during the experiment. For the ADCs of the CDH and the IH, reconstructed pion tracks with the CDC were used as the calibration source, considering their track length in the scintillator event by event. 

\subsection{Time-walk correction and time offset tune for scintillation counters.}
The time-walk effect is well-known phenomenon that the timing signal generated by a threshold-type discriminator has systematic dependency on the pulse height or the charge of the raw signal. The effect was compensated at offline analysis with a correction function,
\begin{eqnarray*}
p_0+\frac{p_1}{\sqrt{dE}}+p_2\cdot dE
\end{eqnarray*}
where $dE$ is the energy deposit on the photo-sensor recorded with an ADC, $p_i$ are the parameters. These parameters were obtained by iteratively correcting the $dE$-timing relation with well defined velocity particles such as $\pi$ beam and $\gamma$-rays. The timing offsets were also adjusted in this procedure by optimizing the parameter $p_0$ run by run.

\subsection{Hit timing determination of a hodoscope segment}
A particle hit on a hodoscope segment was identified by the coincidence of the TDC signals of all photomultipliers; one for the IH, and two for the other detectors. The timing was determined by the mean of the TDC timings after the time-walk correction.