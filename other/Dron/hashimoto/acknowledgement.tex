\chapter*{Acknowlegdement}
First and foremost, I would like to express my sincere gratitude to Prof. Ryugo S. Hayano, who introduced me this interesting world of physics and supported me in these five years as my supervisor. At every difficulties and crossroads, there were his supportive advises and encouragements. I was really impressed by not only his attitude toward physics researches but also his selfless contribution to the society with his expertise.

Secondary, I am deeply grateful to Prof. Masahiko Iwasaki, who is the spokesperson of the present experiment, J-PARC E15. He gave me a chance to participate in the experiment, many advises and encouragements. His sharp insight always guided my analysis. I am also grateful to him for accepting me as a junior research associate at RIKEN in the first year of my Ph.D course. 

I would like to acknowledge Dr. Fuminori Sakuma, who was a on-site responsible person of the present experiment. He was always on negotiation with facility staffs and technical staffs to create better environment, while leaving the experiment itself to my discretion. I also acknowledge his significant contribution in developing an event generator for the Geant4-base Monte Carlo simulation.

Special thanks go to Dr. Masaharu Sato, who I spend the longest time with in these five years. I always tried to follow his way, and I learned most of the basics of the experimental research in the daily work with him. In the experiment, his dedicated contribution on the development and the operation of the liquid helium-3 target was quite remarkable and essential for the successful data taking. For the cryogenic target system, I would like to acknowledge Dr. Masami Iio for the significant contribution in the development of the system and for pleasant days at KEK and J-PARC. I am also grateful to Dr. Shigeru Ishimoto and Mr. Shoji Suzuki for the helps with their rich experience on cryogenic targets. I learned a lot about the cryogenics from these four specialists.

I am grateful to Dr. Haruhiko Outa and Dr. Takatoshi Suzuki, who taught me the basics of the kaon beam experiment. Their convincing comments in the meetings made my ideas and understandings so clear. I am grateful to Dr. Kenta Itahashi for the help in the DAQ work and for providing an analysis server. I am also grateful to Prof. Hiroyuki Noumi for giving me advises and encouragements at the experimental cite.

I would like to thank Dr. Hiroaki Ohnishi for sparing a substantial amount of time for me to discuss about various things, such as detector developments, analysis methods, physics interpretation of our data, and future experimental plans. His positive and practical advices always made me go ahead. I acknowledge Dr. Yue Ma for many helps in the DAQ and simulation works. %Dr. Hiroaki Ohnishi and Dr. Yue Ma sometimes joined ``Hanseikai",  which were held with Dr. Fuminori Sakuma and Dr. Masaharu Sato after every meeting. We talked many topics over many glass of beers and they always encouraged me. It was my important source of motivation to accomplish this work.

I would like to acknowledge Ph.D candidates at the K1.8BR beam line; Mr. Shun Enomoto, Mr. Yuta Sada, Mr. Makoto Tokuda, Mr. Kentaro Inoue, Mr. Zhang Qi,  Mr. Shingo Kawasaki, and Mr. Takumi Yamaga. Without their continuous contributions in the construction of the spectrometer and in the data taking, the present experiment has not been accomplished.

I acknowledge all the collaborators of the J-PARC E15 experiment for their contributions in every stage of the experiment. Especially, I would like to acknowledge those who I had chance to work with in these five years; Prof. Tomofumi Nagae, Prof. Eberhard Widmann, Prof. Johann Zmeskal, Dr. Ken Suzuki, Dr. Tomoichi Ishiwatari, Dr. Shinji Okada, Dr. Kyo Tsukada, Dr. Hiroyuki Fujioka, Dr. Hexi Shi, Dr. Barbara K. Wunschek, Mr. Toshihiko Hiraiwa, Mr. Hiroshi Kou and Mr. Yosuke Ishiguro.

I am also grateful to Prof. Toshimitsu Yamazaki, who motivated us to perform the present experiment with the prediction of the kaonic nuclear states by the theoretical work in collaboration with Prof. Yoshinori Akaishi. Prof. Yamazaki often gave me comments on the experimental analysis and instructed me the basic idea of the kaonic nuclear physics.

The present experiment was deeply embedded in the support by the staffs of J-PARC, especially in the accelerator group and in the Hadron experimental facility. I am really grateful to them for their hard works to deliver higher quality beam to the experimental groups. The quick recovery from the earthquake was really amazing.% and I am really proud of having performed the first charged kaon experiment at J-PARC.

I would like to acknowledge the referees of the present thesis; Prof. Kyoichiro Ozawa (chief), Prof. Hiroyoshi Sakurai, Prof. Tetsuo Hatsuda, Prof. Naohito Saito, and Prof. Osamu Morimatsu. I really appreciate their valuable comments to improve the thesis.

Although I spend most of time at KEK, Tsukuba and at J-PARC, Tokai, my school life was at Hongo with the members of the Nuclear Experimental Group. I would like to express my special thanks to them; Prof. Hideyuki Sakai and Sakai group: Prof. Kentaro Yako, Dr. Shumpei Noji, and Dr. Kenjiro Miki; Prof. Kyoichiro Ozawa and Ozawa group: Mr. Yosuke Watanabe, Mr. Kazuki Utsunomiya, Mr. Shinichi Masumoto, Mr. Yusuke Komatsu, and Ms. Atsuko Takagi; Prof Hiroyoshi Sakurai and Sakurai group: Dr. Megumi Niikura, Dr. Nobuyuki Kobayashi, Mr. Zhengyu Xu, Mr. Keishi Matsui, Mr. Ryo Taniuchi, Mr. Takuya Miyazaki, and Mr. Satoru Momiyama; Hayano group: Dr. Taro Nakao, Dr. Hideyuki Tatsuno, Dr. Satoshi Ito, Mr. Yuya Fujiwara, Mr. Takumi Kobayashi, Mr. Koichi Todoroki, Mr. Takahiro Nishi, Mr. Kouta Okochi, Mr. Yoshiki Tanaka, Mr. Yohei Murakami, Mr. Hiroyuki Yamada, and Mr. Yuni Watanabe. I enjoyed daily discussions and conversations with them, and learned a lot from them. 

Last but not least, I would like to express gratitude to my family and all my friends for their continuous supports and encouragements.


