\chapter{Conclusion}

We have searched for the $K^-pp$ bound state via a missing-mass measurement of the $^3$He($K^-,n$) reaction at the kaon beam momentum of 1 GeV/$c$. The $K^-pp$ state is the lightest kaonic nucleus, whose existence is supported theoretically as a consequence of the strong $\bar KN$($I$=0) attraction but not established experimentally yet. In our employed reaction, the $K^-pp$ signal can be effectively searched for, since physical background processes in the bound region are expected to be almost free due to the in-flight reaction kinematics of $^3$He($K^-,n$). Helium-3 is the simplest target to search for the $K^-pp$ state in the $(K^-,n)$ reaction and the measurement of the $^3$He($K^-,n$) reaction was performed for the first time.

The first-stage experiment was carried out in May, 2013 at the K1.8BR beam line in the J-PARC hadron experimental facility. We radiated 5.2 $\times 10^{9}$ kaons on a helium-3 target during the data taking period of about 4 days. The spectrometer system we constructed for the present experiment consists of a beam line spectrometer, a liquid helium cryogenic target system, a cylindrical detector system (CDS) surrounding the target, and a neutron detection system at the forward angle. They worked as designed and we achieved the missing-mass resolution of $\sim$10 MeV ($\sigma$) around the $K^-pp$ binding threshold (2.37 GeV$/c^2$).
\\

The $^3$He($K^-,n)X$ missing-mass distribution was successfully obtained in a semi-inclusive condition requiring at least one charged track in the CDS, which covers the polar angle range from 54 to 126 degrees. We found quite small yield in the deep bound region from the $K^-pp$ binding threshold, where contributions of non-mesonic two-nucleon absorption processes, $K^-NN\to YN$, should appear if such processes exist in the in-flight $^3$He($K^-,n)$ reaction. On the other hand, we have a certain amount of events just below the threshold. They cannot be explained by experimental effects, such as the detector resolution and unphysical backgrounds, nor known elementary processes. The number of the events in this unknown excess was evaluated to be 1462 $\pm$ 58(stat.) $\pm $122(syst.) in a missing-mass region from 2.29 GeV/$c^2$ to 2.37 GeV/$c^2$, while the background was to be 568 $\pm$ 57(stat.) $\pm$ 121(syst.). Therefore, the unknown excess is statistically significant. Such a structure just below $K^-pp$ threshold was firstly observed by the present experiment.
\\

The possible explanations of the observed unknown excess were discussed. Contributions from mesonic two-nucleon absorption and three-nucleon absorption processes were found to be substantially small. The $\Lambda(1405)n$ branch of non-mesonic two-nucleon absorption reaction can explain the excess with a cross section as large as 5 mb/sr at $\theta_{lab}=0^\circ$, which cannot be distinguished from the shallowly-bound $K^-pp$ state with the present data. In the case that the excess was fully attributed to the $K^-pp$ formation, the cross section at $\theta_{lab}=0^\circ$ was evaluated to be (1.2$\sim$1.6) $\pm$ 0.3 mb/sr, according to the assumed decay modes of $K^-pp\to\Lambda p,\ \Sigma^0p,$ and $\pi\Sigma p$.

For the deep bound region below 2.28 GeV/$c$ in the neutron missing mass spectrum, corresponding to the $K^-pp$ binding energy of larger than 90 MeV, upper limits of the formation cross-section of the deeply-bound $K^-pp$ state were evaluated. They were determined to be 0.02$\sim$0.4 mb/sr at $\theta_{lab}=0^\circ$ at a 95\% confidence level. Here we assumed the widths of 20, 60 and 100 MeV/$c^2$ with a decay mode of $K^-pp \to \Lambda p$. We found no statistically significant structure in the deep bound region claimed by the FINUDA and DISTO groups. Also our results contradict to the theoretical calculation by Koike and Harada, where potentials are obtained from those experimental results.%, and the shape of unknown excess is rather consistent to the shallow binding $K^-pp$ with relatively large width. 
\\

Our setup has a good capability to carry out an exclusive analysis by the coincidence measurement of decay particles and a forward-going particle. Although such analysis needs much more data than we accumulated in the first physics run, it is one of the best ways to derive a conclusive evidence of the existence of the bound $K^-pp$ state. In addition, data with hydrogen and deuterium targets are of vital importance not only for better understanding of the spectrometer system to reduce systematic uncertainties but also for more quantitative discussions of known processes as backgrounds of the $K^-pp$ search.

These data takings will be performed in near future and provide crucial information on the $K^-pp$ state and the $\bar K N$ interaction.