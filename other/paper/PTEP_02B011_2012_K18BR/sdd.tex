The goal of the E17 experiment is to achieve a precision measurement of both kaonic-$^3$He and $^4$He atoms with an unprecedented accuracy of 1 eV. To achieve this precision, we employ Silicon Drift Detectors (SDDs) with a large active area. 

The concept of an SDD was originally introduced by E. Gatti and P. Rehak in 1984 \cite{Gatt84}.
% A partially cutaway view of an SDD is shown in Fig. \ref{sdd:drawing}. 
The electric field parallel to the surface of the detector is generated by ring electrodes biased gradually. 
Electrons created by incoming X-ray absorption are drifted toward a collection anode placed at the center of the detector.
The distinguishing feature of SDDs is the extremely small anode size
which results in low capacitance of the detector.
It is also independent of the detector active area, thus large size of the active area of 100 mm$^2$ becomes possible with low capacitance.
To take advantage of the low output capacitance, an FET for the first-stage amplification is directly integrated on the detector chip. It is connected to the anode with a short metal strip to minimize the stray capacitance and microphonic noise. 
A typical energy resolution of 150~eV is obtained at 6~keV with sufficient noise-reduction. The time resolution is typically sub-micro second below 200 K, which is mainly determined by the drift-time distribution of electrons in silicon. 
In recent years, several types of SDDs with a large active area have been developed. For X-ray spectroscopy of kaonic atoms, SDDs were used in the KEK-PS E570 experiment in KEK\cite{E570}, and the SIDDHARTA experiment in LNF\cite{SIDDHARTA}.

In the J-PARC E17 experiment, 8 SDDs and reset-type preamplifiers developed by KETEK\footnote{KETEK GmbH, Vitus-SDD without window and collimator} were adopted.
Each detector has a thickness of 0.45 mm and an active area of 100 mm$^2$.
They are mounted around the liquid helium target as illustrated in the inset of Fig. \ref{lhe3:cryo}. 
The acceptance for both kaonic $^3$He and $^4$He $L_{\alpha}$ X-rays is approximately 1 \% with 8 SDDs taking into account the attenuation inside the helium target, the target cell, etc. 
The preamplifiers are also installed in vacuum to minimize the cable length between SDDs and preamplifiers. Output signal from the preamplifiers are connected to a CAEN N568b shaping amplifier. Semi-Gaussian outputs are provided with different shaping times of 0.5 and 3.0 $\mu$s.  Output signals with 0.5 $\mu$s shaping time are used for the timing information of the SDDs in coincidence with an incoming $K^-$. Signals with 3.0 $\mu$s shaping time are recorded with two types of peak-hold ADCs, TKO peak-hold ADC and VME CAEN V785. For the purpose of the pileup rejection, the line-shapes of the signals are also recorded by a flash ADC (SIS3301, 14 bit, 105MHz).

To reduce the heat loads on the helium target, SDDs must be  operated at low temperature. %Nevertheless they function well even with simplified Peltier cooling, like the Peltier one.
We performed basic studies on the temperature dependence of the energy- and time- resolutions%\cite{barbara}
,  and optimized the operational temperature of SDDs to 130 K. 
Heat load from the pre-amplifiers is minimized by covering them with the heat shields cooled down to 77 K.
All 8 SDDs were operated successfully inside the cryostat during cooling of the  liquid $^4$He target. 

In November 2010, we performed commissioning of SDDs with the secondary beam at the K1.8BR beamline.  The spectrum obtained with 8 SDDs is shown in Fig. \ref{sdd:spectrum}. 
The absolute energy calibration was performed with $K_{\alpha}$ fluorescence X-rays from titanium (4.5 keV) and nickel (7.5 keV) foils induced by the beam particles.
Furthermore we installed an iron foil at the target position to have fluorescence X-rays at an energy around 6.5 keV. 
As shown in the lower panel of Fig. \ref{sdd:spectrum}, we obtained the energy resolution (FWHM) for three different energies.
The dotted line in the figure is an empirical formula, $2.35 \omega \sqrt{W_N^2 + FE/\omega}$, where $E$, $\omega$, $W_N$,  and $F$ are incident X-ray energy,  electron-hole pair creation energy, noise constant and Fano factor, respectively. 
The energy dependence is well understood by the known formula.
Obtained values are $F = 0.14 \pm 0.01, W_N = 6.7 \pm 0.8$ with a constant of $\omega = 3.81$ eV.
We achieved an energy resolution of 150 eV FWHM at an energy for kaonic $L_\alpha$ lines in $^3$He and $^4$He, which is better than the 185~eV of KEK-PS E570.
This energy resolution is sufficient to achieve a precision of 1 eV with expected statistics of kaonic $L_\alpha$ of 5,000 events.
Signal to noise ratio for the calibration spectrum in the commissioning  is also higher than that of E570 by a factor of 3.  
As a result of the beam commissioning, good performance of the SDDs was achieved in realistic conditions.

%\begin{figure}[h]
%\begin{minipage}{0.45\columnwidth}
%\includegraphics[width=\columnwidth]{sdd-drawing.eps}
%\caption{\label{sdd:drawing} A conceptual drawing of SDD. Electrons generated by initial photons are drifted toward a collection anode placed at the center of the detector.}
%\end{minipage}\hspace{2pc}%
%\begin{minipage}{0.45\columnwidth}
%\includegraphics[width=\columnwidth]{sdd-spectrum.eps}
%\caption{\label{sdd:spectrum} A spectrum with fluorescence x-rays induced by beam at  K1.8BR (Top) and the energy dependence of the energy resolution (Bottom). Dotted line represents an empirical formula described in the text. 
%}
%\end{minipage} 
%\end{figure}

\begin{figure}[t]
\begin{center}
\includegraphics[width=0.5\columnwidth]{sdd-spectrum.eps}
\caption{\label{sdd:spectrum} A spectrum with fluorescence x-rays induced by beam at  K1.8BR (Top) and the energy dependence of the energy resolution (Bottom). Dotted line represents an empirical formula described in the text. 
}
\end{center}
\end{figure}


% References 
% {Gatt84} E. Gatti and P. Rehak, 1984 Nucl. Instr. Meth. A 225(1984)608-614
% {Okad07} S. Okada et al., Phys. Lett. B 653(2007)387-391
% {Bazz09} M. Bazzi et al., Phys. Lett. B 681(2009)310-314
