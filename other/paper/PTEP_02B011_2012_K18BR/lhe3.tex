%\subsection{Liquid $^3$He target system} 
\subsubsection{Configuration and operational procedure} 
A schematic drawing of the liquid $^3$He cryostat is shown in Fig. \ref{lhe3:cryo} for the case of the J-PARC E15 setup. 
The details of the $^3$He target system can be found in a separate paper \cite{Iio12}. 
The major difference between the E15 and E17 settings is the configuration around the target cell. 
To maximize the acceptance for the kaonic helium X-rays in E17,  eight silicon drift detectors (SDDs) will be installed around the target cell as shown
 in the inset of Fig. \ref{lhe3:cryo}. 
In contrast, a time projection chamber will be installed between the target vacuum chamber and the CDS in the E15 setup.  
Thus, the diameter of the vacuum chamber is minimized as much as possible.
The major cryogenic component is divided into three sections; a $^4$He separator,  a $^4$He evaporator and  a heat exchanger between $^3$He and $^4$He.
The target cell is connected to the bottom of the heat exchanger with two 1 meter long pipes.
To reduce the radiation from room temperature components, all low-temperature parts are covered with a radiation shield  anchored to the liquid nitrogen tank (LN$_2$ tank). 

%%operation
The operational concept of the cryostat of E15 and E17 is essentially the same. 
Typical start-up procedure begins with liquid nitrogen cooling. 
When the $^4$He separator and the LN$_2$ tank are filled, the evaporator, the heat exchanger and the target cell are cooled by thermal conduction and radiation.
After the pre-cooling, the liquid nitrogen in the separator is purged, and the liquid helium is transferred from a 1000 $\ell$ Dewar (not shown in the figure) to the separator by a transfer tube. 
The liquid flow is controlled by the pressure inside the separator evacuated by a dry pump. Liquid $^4$He inside the separator is fed to the $^4$He evaporator though a needle valve.
The vapor pressure in the evaporator is reduced by a rotary pump with pumping speed of 120 m$^3$/h, resulting in a heat-removal capability of 2.5 W at 2 K. 
The temperature inside the evaporator is controlled within a range of  1.3 to 2.0 K. This range is mainly determined by the flow rate from the separator to the evaporator. The lowest temperature is achieved with no flow from the separator because the liquid temperature in the separator is fairly high (4.2 K). For liquefaction of $^3$He, the heat exchanger between liquid $^4$He and gaseous $^3$He is positioned below the evaporator. The top part of the heat exchanger, where the liquid $^4$He in the evaporator is in direct contact, has a specially-designed fin structure with both a width and pitch of 0.5 mm. 

A gas-tight handling system (leak rate less than 10$^{-10}$ Pa$\cdot$m$^3$/sec)
 has been constructed to store, transfer and recover the scarce $^3$He gas.
The total amount of 400 $\ell$ of gaseous $^3$He is stored at pressures of less than an atmosphere at room temperature in two 200 $\ell$ tanks. During the cooling stage, those gas tanks are connected to the heat exchanger through the gas handling system.  
 By an effective heat contact inside the heat exchanger, gaseous $^3$He is liquefied, and the liquid $^3$He is flowed to the target cell (6.8~cm in diameter and 13.7~cm in length) through the lower pipe.
 At the last stage of the cooling, most of the $^3$He gas is liquefied inside the target cell and the heat exchanger. 

%%% siphon method %%%
In the L-shaped cryostat, the heat load on the target cell must be transferred effectively to the heat exchanger where the cooling power exists. Otherwise boiling in the target cell occurs.
To accomplish this, we applied the {\it siphon method} as described in Ref. \cite{Iio12}, which uses convection of the liquid $^3$He.
The liquid $^3$He warmed by the heat load inside the target cell returns to the heat exchanger through an upper pipe. In the heat exchanger, $^3$He is cooled again and fed to the target cell through the lower pipe. This makes possible the heat transfer between the target cell and the heat exchanger.  

%%% one shot %%%
For long-term operation, it is essential to reduce the total amount of $^4$He consumed. This is because exchanging the $^4$He Dewar causes significant experimental dead time. 
To minimize the $^4$He consumption, we adopted {\it one-shot} operation. 
This operation consists of two modes;
(I) the evaporator is filled up with liquid $^4$He supplied from the separator. (II) The $^4$He supply is stopped until the evaporator becomes empty. 
The operational procedure consists of a repetition of these two methods, and 
this reduces the total liquid $^4$He consumption due to the minimization of the transfer loss to the cryostat. 
The operational performance of the target system is described in the following.

\begin{figure}[t]
\includegraphics[width=\columnwidth]{lhe3-cryo.eps}
\caption{ \label{lhe3:cryo}
Schematic drawing of the liquid $^3$He cryostat.
}
\end{figure}
%%%%%%%%%%%%%%%%%%%%%%%%%%%%%%%%%%%%%%%%%%%%%%%%%%%%%%%%%%%%%%%%%%%%%%%%%%%%%%

%\subsubsection{$^3$He target cell} 
% Since kaonic helium $L_{\alpha}$ X-rays have an energy of 6 keV, they are easy to be absorbed in the target cell before detection by X-ray detectors. 
%The choice of the material of the target cell is extremely important.
%We have developed the target cell made of pure beryllium. 
%%The detailed configuration is found in Ref. \cite{Iio12}. 
%The target cell has a cylindrical shape of 68 mm diameter and 138 cm long. 
%The side wall is made of pure beryllium of 0.3 mm thick. 
%The volume of the target cell is 0.48 $\ell$ in which the volume of 269 $\ell$ gaseous $^3$He at room temperature is necessary to fill up with liquid at 1.3 K.
%The purity of the beryllium cylinder is more than 99.4 \%, and the transmittance of 5.9 keV X-rays are confirmed to be $85.3\pm 1.0$ \% by manganese  $K_{\alpha}$ X-rays from radioactive $^{55}$Fe source. 
%This is comparable with that of pure beryllium of 87.2 \%.

%\subsubsection{Vacuum chamber} 
%A vacuum chamber was specially designed to reduce multiple scattering of secondary charged particles.
%As a result of the development in KEK \cite{Sato09}, the material of the vacuum chamber was carefully selected to Carbon Fiber Reinforced Plastic (CFRP) in the region covering the acceptance of the CDS. 
%The detail of the specification of adopted CFRP is found in Ref. \cite{Iio12}.
%Total thickness of the CFRP in the E15 setup is 1 mm with the diameter of 150 mm and the length of 523 mm. 
% An aluminum beam window with the thickness of 0.6 mm is glued to the CFRP with STYCAST 1266.
%A safety factor of this vacuum chamber is estimated to be 3.8 against the external pressure of 1 atm. 

\subsubsection{Operation and performance} 
Along with the operational procedure previously described, cooling tests were performed. 
After the $^4$He transfer, it took about 2 - 3 hours to liquefy  the $^3$He gas in the heat exchanger, achieving thermal equilibrium within 6 hours and a temperature of 1.30 $\pm$ 0.01 K in the target cell without flow from the separator (mode (II) in the {\it one-shot} operation).
The liquid $^3$He density at this temperature is 0.0812~g/cm$^3$ corresponding to the thickness of 1.11~g/cm$^2$.
The density fluctuation due to the temperature instability is less than 0.1 \%.
The temperature differences among the evaporator, the heat exchanger and the target cell are less than 0.01 K. This means the heat transfer by {\it siphon method} is working well.
Furthermore, the pressure inside the heat exchanger was identical to the vapor pressure of liquid $^3$He at the corresponding temperature. Taking into consideration the remaining pressure inside the tanks, a total amount of 380 $\ell$ was condensed, giving evidence that sufficient $^3$He gas is liquefied to fill the target.

From the reduction rate of the liquid $^4$He in the evaporator, 
the heat load of the low-temperature region was estimated to be 0.21 W with the E15 setting.
In the E17 setting, the heat load was expected to increase due to the radiation from the SDDs to the target. 
It was measured to be 0.39 W with the actual E17 setting, and both of them are acceptably small for long-term operation.
The  operational result of the cryostat with the E15 setting is tabulated below.

\begin{table}[h] 
\begin{center}
\caption{\label{operation}Operational result.}
\begin{tabular}{lrc}\hline\hline
vacuum level                      & [mbar]                 &$ < 10^{-6}$ \\ 
leak rate of the $^3$He system    & [Pa$\cdot$m$^3$/sec] &$< 10^{-10}$ \\ 
temperature in the target cell    &[K]       & 1.3        \\
vapor pressure in the target            &[mbar]    & 33         \\
heat load to low-temperature part   &[W]       & 0.21       \\ 
liquid $^4$He consumption          &($\ell$/day)      & 50         \\ 
 \hline \hline  
\end{tabular}
\end{center}
\end{table}
Finally, we note that this cryostat can be utilized as a liquid $^4$He target system by liquefying gaseous $^4$He instead of $^3$He. The operational procedure and the performance of the liquid $^4$He target are the same as those of $^3$He.
The density of the liquid $^4$He is 0.145~g/cm$^3$ at 1.3~K with a stability of better than 0.1 \%, and the thickness is 1.99 g/cm$^2$. 



%%%  References
% M. Iio et al., submitted to Nucl. Instr. and Meth. A
% M. Sato et al., Nucl. Instr. and Meth. A 606 (2009) 233 
