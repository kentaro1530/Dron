%\subsection{Liquid D$_2$ target system} 
For a spectroscopic study of $\Lambda$(1405) by the $d(K^-,n)$ reaction (J-PARC E31), we have been developing a liquid D$_2$ target system.
A side-view of the cryostat is shown in Fig. \ref{d2:cryo}. Since we measure the decay products of $\Lambda$(1405), the target cell, whose size is 6.8~cm in diameter and 12.5~cm in length, is isolated at the center of the CDS in the same way as the liquid $^3$He cryostat. 
The major difference from the  liquid $^3$He target is that this cryostat is coolant-free. The key component of the system is a two-stage Gifford-McMahon (G-M) refrigerator (Sumitomo Heavy Industries, Ltd.  RDK-145D and CSA-71A) built in the cryostat.
% G-M
 The cooling power at the first and second stages is 35 W at 50 K and 1.5 W at 4.2 K, respectively. 
As it is for the $^3$He target, a gas handling system has been constructed for the D$_2$ target. 
The 1000 $\ell$ of gaseous D$_2$ is stored in a tank at 2 bar at room temperature.  
To avoid contamination, the amount of D$_2$ gas was chosen to maintain positive pressure inside the gas system even after liquefaction in the target.


The D$_2$ gas is fed into the  cryostat through the top flange.
For pre-cooling, the inlet pipe for the D$_2$ gas is anchored to a copper plate attached to the first-stage cold head of the G-M refrigerator. 
Another inlet pipe is directly connected through the top flange to the heat exchanger. It is used to measure the D$_2$ pressure inside the heat exchanger. 
Since this pipe has a larger conductance, a safety valve which prevents a sudden pressure rise is also connected to it.
The D$_2$ gas is cooled in the heat exchanger where the second stage of the G-M refrigerator is thermally contacted. 
The structure of the heat exchanger is similar to that of the $^3$He system \cite{Iio12}.
The main difficulty in the operation of the system is the precise control of the temperature in the heat exchanger. 
In the liquid D$_2$ target system, the {\it siphon method}, described previously, is also adopted. 
For effective heat transfer between the target cell and the heat exchanger, D$_2$ must be kept in a liquid state by controlling the temperature to avoid blocking of pipes by solid D$_2$. 
The temperature range of liquid D$_2$ is 18.7-23.8 K at 1 bar, thus the
temperature should be kept around 20 K within acceptable limits.
The thickness of D$_2$ along the beam is 2.13~g/cm$^2$ with the density of 0.17~g/cm$^3$ at 20 K.
Since the cooling power of the second stage of the G-M refrigerator is larger than the heat load on the low-temperature parts, we have installed a heater near the cold finger to compensate the heat load. 
The current in the heater is controlled by a proportional-integral-derivative (PID) algorithm 
 with an input of the temperature of the heat exchanger.

Until now, 
we have installed the G-M refrigerator and heat exchanger in the vertical part of the cryostat. Using hydrogen gas for convenience, 
preliminary tests were performed to confirm the temperature stability.  
In these tests, the target cell was directly connected to the heat exchanger.
The result shows that the temperature of the heat exchanger is controlled within 0.2 K, which is sufficiently stable for the operation. 
By the end of 2012, cooling tests will be performed on the final setup with the D$_2$ gas.

\begin{figure}[t]
\includegraphics[width=\columnwidth]{d2-cryo.eps}
\caption{ \label{d2:cryo}
Schematic drawing of the liquid D$_2$ cryostat.}
\end{figure}
