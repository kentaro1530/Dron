\subsection{Fit demonstration for contributions of $I=0$, $I=1$, and their interference term}
\begin{figure}[htbp]
  \begin{tabular}{cc}
    \begin{minipage}{0.5\hsize}
      \centering
      \includegraphics[width=6.0cm]{../pic/Dron/fit_model_A_fix_Phi/I0_fit.eps}
    \end{minipage}

    \begin{minipage}{0.5\hsize}
      \centering
      \includegraphics[width=6.0cm]{../pic/Dron/fit_model_A_fix_Phi/pimS0_fit.eps}
    \end{minipage}
  \end{tabular}

  \centering
  \includegraphics[width=6.0cm]{../pic/Dron/fit_model_A_fix_Phi/interfer_fit.eps}
  \caption{
    These figures show the fitting of parametric strengths of $I=0$ and $I=1$ using Model.A.
    The notation is the same as in Fig.\ref{fig:decomposed_DCC}.
    Only the red line representing Model.A (red line) is plotted.
  }
  \label{fig:fit_A_scale}
\end{figure}


First, free parameterize the strength of the second scattering of isospin $I=0,1$ as $\Tmat'^{I=0, 1} = A_{I=0,1}\Tmat^{I=0,1}$, where $A_{I=0,1}$ is the free parameters.

Here, the scattering strength of $I=1,0$ and its interference term are expressed as
\begin{align}
  & f_{I=0}(m_{\pi\Sigma})=\left| \Cfirst^0 \Tmat^{I=0} \right|^2 \\
  & f_{I=1}(m_{\pi\Sigma})=\left| \Cfirst^1 \Tmat^{I=1} \right|^2 \\
  & f_{int}(m_{\pi\Sigma})=2\mbox{Re}( \Cfirst^0 \Cfirst^1 \Tmat^{I=0} \Tmat^{I=1} ) \label{eq:interfer}
\end{align}

Eq.(\ref{eq:Charge_piS}, \ref{eq:pimS0}) can be expressed as

\begin{align}
  \frac{d\sigma}{d\Omega dM}(\pi^\mp \Sigma^\pm) = & A_{I=0} f_{I=0}(m_{\pi\Sigma}) + A_{I=1} f_{I=1}(m_{\pi\Sigma}) \nonumber \\
                                                   & + \sqrt{A_{I=0}A_{I=1}} f_{int}(m_{\pi\Sigma}) \label{eq:scale_Charge} \\
  \frac{d\sigma}{d\Omega dM}(\pi^- \Sigma^0) = & A_{I=1} f_{I=1}(m_{\pi\Sigma}) \label{eq:scale_pimS0} 
\end{align}

\newcommand{\fitScatLength}{-1.05 \pm 0.12 (fit.) \pm +0.09 (syst.) + [ 0.86 \pm 0.15 (fit.) ^{+0.07} _{-0.08} (syst.)]i}
\newcommand{\fitEffRange}{-0.22 \pm 0.40 (fit.) ^{+0.05} _{-0.06} (syst.) + [ -0.42 \pm 0.16 (fit.) ^{+0.12} _{-0.08} (syst.)]i}
\newcommand{\fitPole}{1418.3 ^{+7.5} _{-2.4} (fit.)^{+0.9}_{-1.1} (syst.)}
\newcommand{\fitWidth}{-27.8^{+9.5}_{-0.9} (fit.)^{+1.9}_{-2.1} (syst.)}

\newcommand{\fitAscaleIz}{0.562 \pm 0.015}
\newcommand{\fitAscaleIo}{1.070 \pm 0.040}
\newcommand{\fitAscaleIzVal}{0.562}
\newcommand{\fitAscaleIzErr}{0.015}
\newcommand{\fitAscaleIoVal}{1.070}
\newcommand{\fitAscaleIoErr}{0.040}
\newcommand{\fitAscaleChi}{691/42}
\newcommand{\fitAscaleChiNum}{16.4}

\newcommand{\fitBscaleIz}{0.721 \pm 0.016}
\newcommand{\fitBscaleIo}{1.423 \pm 0.055}
\newcommand{\fitBscaleIzVal}{0.721}
\newcommand{\fitBscaleIoVal}{1.423}
\newcommand{\fitBscaleIzErr}{0.016}
\newcommand{\fitBscaleIoErr}{0.055}
\newcommand{\fitBscaleChi}{220/42}
\newcommand{\fitBscaleChiNum}{5.25}

\newcommand{\fitBBChi}{187/41}
\newcommand{\fitBBChiNum}{4.56}
\newcommand{\fitBBIz}{0.686 \pm 0.017}
\newcommand{\fitBBIo}{1.462 \pm 0.059}
\newcommand{\fitBBphase}{0.828 \pm 0.030}
\newcommand{\fitBBIzVal}{0.686}
\newcommand{\fitBBIoVal}{1.462}
\newcommand{\fitBBphaseVal}{0.828}
\newcommand{\fitBBIzErr}{0.017}
\newcommand{\fitBBIoErr}{0.059}
\newcommand{\fitBBphaseErr}{0.030}
\newcommand{\fitBBDegree}{34.1^{+3.0}_{-3.2}}

\newcommand{\fitBChi}{184/41}
\newcommand{\fitBChiNum}{4.48}
\newcommand{\fitBIz}{0.682 \pm 0.017}
\newcommand{\fitBIo}{1.570 \pm 0.058}
\newcommand{\fitBphase}{0.811 \pm 0.030}
\newcommand{\fitBIzVal}{0.682}
\newcommand{\fitBIoVal}{1.570}
\newcommand{\fitBphaseVal}{0.811}
\newcommand{\fitBIzErr}{0.017}
\newcommand{\fitBIoErr}{0.058}
\newcommand{\fitBphaseErr}{0.030}
\newcommand{\fitBDegree}{35.8^{+2.8}_{-3.1}}

\newcommand{\fitScaleKN}{0.0372 \pm 0.0047}
\newcommand{\fitAreKN}{-1.05 \pm 0.12}
\newcommand{\fitAimKN}{ 0.86 \pm 0.15}
\newcommand{\fitRreKN}{-0.22 \pm 0.40}
\newcommand{\fitRimKN}{ 0.42 \pm 0.16}
\newcommand{\fitKNMass}{1418.3}
\newcommand{\fitKNWidth}{27.8}

\newcommand{\fitScaleKmp}{0.0377 \pm 0.0042}
\newcommand{\fitAreKmp}{-0.95 \pm 0.11}
\newcommand{\fitAimKmp}{ 0.94 \pm 0.16}
\newcommand{\fitRreKmp}{-0.27 \pm 0.40}
\newcommand{\fitRimKmp}{ 0.52 \pm 0.18}
\newcommand{\fitKmpMass}{1417.6}
\newcommand{\fitKmpWidth}{30.3}

\newcommand{\fitScaleKzeroN}{0.0367 \pm 0.0053}
\newcommand{\fitAreKzeroN}{-1.13 \pm 0.13}
\newcommand{\fitAimKzeroN}{ 0.79 \pm 0.15}
\newcommand{\fitRreKzeroN}{-0.16 \pm 0.40}
\newcommand{\fitRimKzeroN}{ 0.33 \pm 0.16}
\newcommand{\fitKzeroNMass}{1419.3}
\newcommand{\fitKzeroNWidth}{25.9}


The results of fitting our spectra by varying the model.A by parameterization according to Eq.(\ref{eq:scale_Charge}-\ref{eq:scale_pimS0}) are shown in Fig.\ref{fig:fit_A_scale}.
where $A_{I=0}$ is $\fitAscaleIz$ and $A_{I=1}$ is $\fitAscaleIo$.
$I=0$ suppresses the overall strength due to the long tail below the $\bar{K}N$ threshold, but still does not explain the overall spectral shape.
The strength at $I=1$ remains almost unchanged, while the strength at $I=0$ is suppressed by the long tail below the $\bar{K}N$ threshold.
Therefore, the strength of the interference term is also smaller and deviates from the experimental value.
The $\chi^2/NDF$ for this fit is $\fitAscaleChi = \fitAscaleChiNum$.

\begin{figure}[htbp]
  \begin{tabular}{cc}
    \begin{minipage}{0.5\hsize}
      \centering
      \includegraphics[width=6.0cm]{../pic/Dron/fit_model_B_fix_Phi/I0_fit.eps}
    \end{minipage}

    \begin{minipage}{0.5\hsize}
      \centering
      \includegraphics[width=6.0cm]{../pic/Dron/fit_model_B_fix_Phi/pimS0_fit.eps}
    \end{minipage}
  \end{tabular}

  \centering
  \includegraphics[width=6.0cm]{../pic/Dron/fit_model_B_fix_Phi/interfer_fit.eps}
  \caption{
    This figure presents the results of fitting the experimental data using Model B.
    The blue line represents the fit results, while other notations follow the same conventions as in Figure \ref{fig:fit_A_scale}.
  }
  \label{fig:fit_B_scale}
\end{figure}


\begin{figure}[htbp]
  \begin{tabular}{cc}
    \begin{minipage}{0.5\hsize}
      \centering
      \includegraphics[width=6.0cm]{../pic/Dron/fit_model_B_2/I0_fit.eps}
    \end{minipage}

    \begin{minipage}{0.5\hsize}
      \centering
      \includegraphics[width=6.0cm]{../pic/Dron/fit_model_B_2/pimS0_fit.eps}
    \end{minipage}
  \end{tabular}

  \centering
  \includegraphics[width=6.0cm]{../pic/Dron/fit_model_B_2/interfer_fit.eps}
  \caption{
    This figure shows the fitting results obtained when the degrees of freedom associated with the interference term
    are introduced as additional fitting parameters.
    The upper-left panel displays the $I=0$ component, the upper-right panel shows the $I=1$ component,
    and the lower panel presents the spectrum of the interference term.
    The error bars denote the experimental data, while the blue curve represents the fit obtained using DCC Model B.
    In this figure, the $I=1$ strength is determind by the only $\pi^- \Sigma^0$ spectrum.
  }
  \label{fig:fit_B_phase_fixI1}
\end{figure}


\begin{figure}[htbp]
  \begin{tabular}{cc}
    \begin{minipage}{0.5\hsize}
      \centering
      \includegraphics[width=6.0cm]{../pic/Dron/fit_model_B/I0_fit.eps}
    \end{minipage}

    \begin{minipage}{0.5\hsize}
      \centering
      \includegraphics[width=6.0cm]{../pic/Dron/fit_model_B/pimS0_fit.eps}
    \end{minipage}
  \end{tabular}

  \centering
  \includegraphics[width=6.0cm]{../pic/Dron/fit_model_B/interfer_fit.eps}
  \caption{
    This figure shows the results of the fitting using Model.B with the introduction of parameters related to the interference term.
    The notation is the same as in Fig.\ref{fig:fit_B_scale}.
    All three parameters are determined simultaneously in this fitting.
  }
  \label{fig:fit_B_phase}
\end{figure}


The results of the fit using Model.B are shown in Figure.\ref{fig:fit_B_scale}.
The parameters in this fit are $A_{I=0}=\fitBscaleIz$ and $A_{I=1}=\fitBscaleIo$.
And, the $\chi^2/NDF$ of this fit is $\fitBscaleChi = \fitBscaleChiNum$, which explains our spectra better than the model.A.
In this fit, the theoretical value of $I=1$ is stronger and the strength of $I=0$ is weaker.
Since each value is constrained by the interference term, each spectrum appears to be inadequate in the region above the $\bar{K}N$ threshold.
Therefore, we introduce a parameter, $A_{int}$, which allows us to move the interference term independently, as follows.

\begin{align}
  \frac{d\sigma}{d\Omega dM}(\pi^\mp \Sigma^\pm) = & A_{I=0} f_{I=0}(m_{\pi\Sigma}) + A_{I=1} f_{I=1}(m_{\pi\Sigma}) \nonumber \\
  & + \sqrt{A_{I=0}A_{I=1}} A_{int} f_{int}(m_{\pi\Sigma}) \label{eq:scale_Charge_2} \\
  \frac{d\sigma}{d\Omega dM}(\pi^- \Sigma^0) = & A_{I=1} f_{I=1}(m_{\pi\Sigma}) \label{eq:scale_pimS0_2} 
\end{align}

We first consider physical mean of the $A_{int}$ parameter.
The interference terms are expressed as in Eq(\ref{eq:interfer}).
The phase difference between the terms for each isospin, $\Cfirst^{I=0,1} \Tmat^{I=0,1}$ write $\delta \theta$ and the equation can be rewrite as follow.

\begin{equation}
  f_{int}(m_{\pi\Sigma})=2\left| \Cfirst^0 \Cfirst^1 \Tmat^{I=0} \Tmat^{I=1} \right| \cos \delta \theta \label{eq:interfer}
\end{equation}

In general, the phase of the scattering amplitude depends on the total energy, but in the first step of $K^-p \rightarrow \bar{K}N$ scattering,
the total energy is determined by the $^K-$ beam and is independent of the $\pi \Sigma$ mass, which is considered reasonable.
Moving the interference terms independently is equivalent to adding to this phase difference a constant parameter independent of the $\pi \Sigma$ mass.

This parameter, which moves the interference terms independently, only affects the $\pi^-\Sigma^+$ and $\pi^+\Sigma^-$ modes,
while the $\pi^-\Sigma^0$ mode is determined only by the $I=1$ strength.
Therefore, we obtained the strength of $I=1$ from the $\pi^- \Sigma^0$ spectrum in the first step
and the strength of $I=0$ and the parameter about the interference term  by the fitting of the $\pi^- \Sigma^+$ and the $\pi^+ \Sigma^-$ spectra in the next step.
As a result, the spectra is obtained as shown in Fig.\ref{fig:fit_B_phase_fixI1} and each parameter is obtained as $A_{I=0}=\fitBBIz$, $A_{I=1}=\fitBBIo$, and $A_{int}=\fitBBphase$.
Evaluating this fitting with equal weights for all spectra as before the introduction of the interference term gives $\chi^2/NDF$ as $\fitBBChi = \fitBBChiNum$,
which is an improvement over before the interference term is introduced.

We also performed simultaneous fitting of all these parameters in all spectra.
The $\chi^2/NDF$ of this fit is $\fitBChi = \fitBChiNum$, which is almost the same value as when the strength of $I=1$ is fixed.
As a result, the spectra is obtained as shown in Fig.\ref{fig:fit_B_phase} and each parameter is obtained as $A_{I=0}=\fitBIz$, $A_{I=1}=\fitBIo$, and $A_{int}=\fitBphase$.
% To do  %% 
The values of these parameters are very close to those of the case with fixed the strength of $I=1$, and these differences may be considered as systematic errors.
Therefore, these parameters are estimated to $A_{I=0}=0.684 \pm 0.002\mbox{(Systematic)} \pm 0.017\mbox{(fitting)}$,
$A_{I=1}=1.516 \pm 0.054 \mbox{(Systematic)} ^{+0.058} _{-0.059} \mbox{(fitting)}$, and
$A_{int}=0.820 \pm 0.09 \mbox{(Systematic)} \pm 0.03 \mbox{(fitting)}$.
That implies adding a constant offset of $34.9 \pm 0.9 \mbox{(Systematic)} \pm 0.3 \mbox{(fitting)}$ degrees the phase difference between $I=0$ and $I=1$.

%% \begin{table}
%%   \begin{tabular}{c|c|c|c}
%%     Scale      & $\fitBscaleIz$ & $\fitBscaleIo$ & - \\
%%     Fix $I=1$  & $\fitBBIz$ & $\fitBBIo$ & $\fitBBphase$ \\
%%     Phase      & $\fitBIz$ & $\fitBIo$ & $\fitBphase$ \\
%%   \end{tabular}
%% \end{table}
