\section{Forward Detector System}
The forward detector system consists of a neutron detector system for detecting forward scattered neutrons
and a proton detector system for detecting protons, although some detectors are used for both purposes.

Downstream of the CDS is a beam sweep magnet, called Ushiwaka, which bends the negatively charged beam into a beam dump.
Positively charged particles are bent to the opposite side by the Ushiwaka, while neutrals fly straight.
The neutron counter (NC) is installed about 15 m downstream of the CDS to detect the neutral particles,
and the charge veto counter (CVC) is installed immediately upstream to confirm that they are not actually charged particles.

The Proton counter (PC) is installed in the opposite direction to the beam dump to detect positive charged particles.
The forward drift chamber (FDC) is installed upstream of the Ushiwaka to detect the position of positive charged particles before they are bent,
whose information is used to momentum analysis and particle identification.

The beam veto counter (BVC) is also installed immediately upstream of the FDC to detect beams that have passed through without reacting with the target, which is described.

\subsection{Beam Sweeping Magnet}
The beam sweep magnets, called Ushiwaka, are located 150 cm downstream of the target and have a pole length of 70 cm.
The aperture of the Ushiwaka is 82cm (horizontal) $\times$ 40cm (vertical), which is wider than the area covered by the NC in the present experimental setup.
The Ushiwaka can be applied to 1.6T at maximum value and is applied to 1.2T in the present experiment.


\subsection{Neutron Counter - NC}
The neutron counter (NC) are located 14.7 m upstream of the target.
Because the NC are located at the most upstream and the purpose is to detecting neutral particles, the NC requires a large amount of material.
Therefore, the NC is a segmented scintillation detector with 7 layers, each layer consisting of 16 counters.
One scintillation detector is 20cm (width) $\times$ 150cm (height) $\times$ 5cm (thickness) in size.
So one layer covers 320cm (width) $\times$ 150cm (height), which is corresponds 6.2 degrees in horizontal and 2.9 degrees in vertical in this experiment setup.
The first three layers of the scintillator are made of Saint-Gobain BC408 and the other four layers are made of Saint-Gobain BC412.
The scintillation light is carried by lucite light guides on both sides and read out by the 2-inch PMT (Hamamatsu H6410).
Each layer is installed with a gap of 2 cm, so the whole NC has a thickness of 47 cm.
Differences between the upstream and downstream solid angles are evaluated as systematic errors.

\subsection{Charge Veto Counter - CVC}
The charge veto counter (CVC) is installed just upstream of the NC to confirm that particles detected by the NC is actually neutral.
Half of the CVC is used to detect the proton.
The CVC consists of 34 segmented plastic scintillators whose size is 10cm (width) $\times$ 150cm (height) $\times$ 5cm (thickness).
A 2-inch Hamamatsu Photomultipliyer H6410 is attached to each side of the scintillator through a Lucite light guide.
The scintillator is of Eljen EJ-200 type. The average time resolution measured with cosmic rays is 78 $\pm$ 7 ps ($\sigma$).
The error represents the variation among the segments.

\subsection{Proton Counter - PC}
The proton detector (PC) is a hodoscope array for detecting positive charged particles located immediately next to the CVC in the direction opposite to the beam dump.
The PC is a 27-segmented detector using scintillators of the same size as the CVC, i.e. it has an effective area of 270 m (horizontal) $\times$ 150 m (vertical).
% However, charged particles are bent and their bending angle depends on their momentum, so their acceptance depends on their momentum.
The photomultipliyer of the PC is the same as the CVC, but the scintillator is a Saint-Gobian BC408.
The average time resolution measured with cosmic rays is 75 $\pm$ 6 ps ($\sigma$),  i.e. the time resolution does not differ from the CVC


\subsection{Forward Drift Chamber - FDC}
\begin{table}
  \caption{Summary of the beam-line chamber parameters.}
  \centering
  \begin{tabular}{l|cccc}
    \hline\hline
    &       BLC1a   &       BLC1b   &       BLC2a   &       BLC2b  \\
    \hline
    number of planes        &       8       &       8       &       8       &       8      \\
    plane configuration     &       \footnotesize{UU'VV'UU'VV'}     &       \footnotesize{UU'VV'UU'VV'}     &       \footnotesize{UU'VV'UU'VV'}     &       \footnotesize{VV'UU'VV'UU'} \\
    \shortstack{number of sense wires \\ in a plane}        &       32      &       32      &       32      &       32     \\
    wire spacing (mm)       &       4       &       4       &       2.5     &       2.5     \\
    effective area (mm)     &       256 $\times$ 256        &       256 $\times$ 256        &       160 $\times$ 160        &       160 $\times$ 160       \\
    \hline
    Sense wire              &               &               &                               &     \\
    ~~material              &       \multicolumn{4}{c}{Au-plated W (3\% Re)}                     \\
    ~~diameter ($\mu m$)    &       \multicolumn{4}{c}{ 12}                                      \\
    \hline
    Potential wire          &               &               &                               &     \\ 
    ~~material              &       \multicolumn{4}{c}{Au-plated Cu-Be}                          \\
    ~~diameter ($\mu m$)    &       \multicolumn{4}{c}{ 75}                                      \\
    \hline
    Cathode plane           &              &                &               &   \\
    ~~material              &       \multicolumn{4}{c}{alminized-Kapton}       \\
    ~~thickness ($\mu m$)   &       \multicolumn{4}{c}{12.5}                   \\
    %% \hline
    %% Gas             &       \multicolumn{4}{c}{ Ar : isoC$_4$H$_{10}$ : Metylal = 76 : 20 :4}   \\
    %% flow (cc/min)   &       \multicolumn{4}{c}{100}                                             \\
    %%    %% %plane efficiency       &       $>$ 99\%        &       $>$ 99\%        &       $>$ 99\%        &       $>$ 99\%        &       $>$ 99\%        &       $>$ 99\%        \
    \hline\hline
    operation voltage       &               &               &               &               \\
    potential               &       -1.25   &       -1.25   &       -1.25   &       -1.25   \\
    cathord                 &       -1.25   &       -1.25   &       -1.25   &       -1.25   \\
    %guard  &       -       &       -       &       -       &       -       &       -       &       -1.85   \\
    \hline\hline
  \end{tabular}

  \begin{tabular}{l|ccc}
    \hline\hline
    &       BPC(Run68)   &  BPC(Run78)   &  FDC1 \\
    \hline
    number of planes        &       8       &       8       &       6     \\
    plane configuration     &       \footnotesize{XX'YY'X'XY'Y}     &       \footnotesize{XX'YY'X'XY'Y}     &       \footnotesize{UU'XX'VV'}  \\
    \shortstack{number of sense wires \\ in a plane}        &       15      &       32      &       64      \\
    wire spacing (mm)       &       3.6       &       3       &     3           \\
    effective area (mm)     &       111.6$\phi$       &  189$\phi$        &      \\
    \hline
    Sense wire              &               &                                &            \\
    ~~material              &       \multicolumn{3}{c}{Au-plated W (3\% Re)}                   \\
    ~~diameter ($\mu m$)    &       \multicolumn{2}{c}{ 12}                  &      12   \\
    \hline
    Potential wire          &                            &               &               \\ 
    ~~material              &       \multicolumn{2}{c}{Au-plated Cu-Be}  &  Au-plated Cu-Be \\
    ~~diameter ($\mu m$)    &       \multicolumn{2}{c}{ 75}              &   75      \\
    \hline
    Cathode plane           &                    &                &             \\
    ~~material              &       \multicolumn{2}{c}{Cu aramid} & alminaized-Kapton \\
    ~~thickness ($\mu m$)   &       \multicolumn{2}{c}{9}         & 7.5     \\
    %% \hline
    %% Gas             &       \multicolumn{4}{c}{ Ar : isoC$_4$H$_{10}$ : Metylal = 76 : 20 :4}   \\
    %% flow (cc/min)   &       \multicolumn{4}{c}{100}                                             \\
    %% %% %plane efficiency       &       $>$ 99\%        &       $>$ 99\%        &       $>$ 99\%        &       $>$ 99\%        &       $>$ 99\%        &       $>$ 99\%        \
    \hline\hline
    operation voltage       &               &               &                      \\
    potential               &       -1.50   &       -1.45   &       -1.45          \\
    cathord                 &       -1.50   &       -1.45   &       -1.45          \\
    %guard  &       -       &       -       &       -       &       -       &       -       &       -1.85   \\
    \hline\hline
  \end{tabular}

  \label{tab:chamber}
\end{table}

For forward scattered positive particles,
The forward drift chamber (FDC) is installed immediately upstream of the Ushiwaka to measure the trajectory of the charged particles before they are bent for momentum analysis.
The information is used for momentum analysis of forward-scattered positive particles for particle identification.

Each pairs are separated by a 20 mm gap using aluminised-Kapton.
The sense wires are arranged with a 6 mm gap and the electric field is created by potential wires surrounding them with squares.
The sense wires and potential wires are the same wires used in the BLC and the BPC.
% The FDC1 has a 400mm (horizontal) $times$ 300mm (vertival) open window, so its effective erea is 363mm (horizontal) $times$ 300mm (vertical) when evaluated as a rectangle.
The configuration of the BLC1/2 and the BPC and the FDC is summarized in Table.\ref{tab:chamber}.
These drift chambers is used same gas.

\subsection{Beam Veto Counter - BVC}
\subsection{Beam Veto Counter - BVC}
\subsection{Beam Veto Counter - BVC}
\input{experimental/figs/BVC}
The BVC is installed immediately upstream of the FDC to avoid beams that did not react with the target.
Its signal is mainly used for forward neutron triggering.
The BVC are installed as horizontal counters with various width due to the vertical dependence of the particle density.
The coverage size of the BVC is 320mm (height) $\times$ 320mm (width) with 10 (thickness) of scintillators, as shwon in Table.\ref{fig:BVC}.
The scintillator is used ELJEN EJ-200.


The BVC is installed immediately upstream of the FDC to avoid beams that did not react with the target.
Its signal is mainly used for forward neutron triggering.
The BVC are installed as horizontal counters with various width due to the vertical dependence of the particle density.
The coverage size of the BVC is 320mm (height) $\times$ 320mm (width) with 10 (thickness) of scintillators, as shwon in Table.\ref{fig:BVC}.
The scintillator is used ELJEN EJ-200.


The BVC is installed immediately upstream of the FDC to avoid beams that did not react with the target.
Its signal is mainly used for forward neutron triggering.
The BVC are installed as horizontal counters with various width due to the vertical dependence of the particle density.
The coverage size of the BVC is 320mm (height) $\times$ 320mm (width) with 10 (thickness) of scintillators, as shwon in Table.\ref{fig:BVC}.
The scintillator is used ELJEN EJ-200.


