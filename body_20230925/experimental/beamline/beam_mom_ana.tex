\subsection{Beam momentum analyzer}
Beam momentum is reconstructed from the second-order transfer matrix of the D5 magnet,
whose magnetic field is monitored during the experiment with high-precision hall probe Lakeshore 475 which has $\sim 10^{-5}$ T resolution.
The fluctuation of the magnetic field was $\sim 2 \times 10^{-4}$ that correspond to $0.2$GeV$/c$ for $1$GeV$/c$ beam.
Also, helium bag was install at pass way of the D5 magnet to suppress the multiple scattering effect due to materials.

For measuring tracks of upstream and downstream of the D5 magnet, planer type drift chambers are installed at these positions.
These chambers are named the BLC1 and the BLC2 which has two components, BLC1a/b and BLC2a/b, respectively.
These chamber sets are rotated 45 degrees with respect to the beam direction axis.
All components have the $UU'VV'UU'VV'$ configuration and 32 sense wires pre layer, so each component has 256 readout channels.
Drift lengths of the BLC1 and the BLC2 are 4mm and 2.5mm which corresponds to cover 252mm $\times$ 252mm and 157.5mm $\times$ 157.5mm area, respectively.
These drift chambers use 12.5  $\mu$m diameter gold-plated tungsten wires with 3\% rhenium for sense wires and 75 $\mu$m diameter copper-beryllium wires for potential wires.
The cathode planes made of 12.5 $\mu$m aluminized Kapton.
The spatial resolution is evaluated at about 200 $\mu$m which is estimated at $0.01\%$ of beam momentum corresponding to $1GeV/c$ for $0.1GeV/c$ beam.
These drift chambers use isobutane and argon including methylal (dimethoxy-methane) to pass through its bubbler
whose temperature is 4 $^{\circ}$C in the refrigerator mixed to 1:4 by the mass-flow controller.
As a result, the ratio of isobutane, argon and methylal was 20\%, 76\% and 4\%.
