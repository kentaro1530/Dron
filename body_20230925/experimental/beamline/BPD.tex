\subsection{Backward Proton Detector - BPD}
BPD which means backward proton detector was developed for the measurement of backward proton decay
from $\Lambda$ in $Y^*\rightarrow \pi^0 \Sigma^0 \rightarrow \pi^0 \gamma \pi^0 \Lambda$ decay scheme which was installed at most upstream of the CDS.
The scattered angle of $Y^*$ is enhanced backward scattering, especially below the threshold due to small momentum transfer.
The BPD was installed to obtain large acceptance for these protons.
The BPD is a plastic scintillator hodoscope array with a size of 350mm (horizontal) $\times$  340mm (vertical).
It is segmented into 70 units of 5mm $\times$ 5mm $\times$ 340mm scintillation counter made of Eljen EJ-230.
Two MPPCs with a 3mm$\times$3mmm sensitive area (Hamamatsu S10362-33-050C) were directly put on both sides of each slab.
The present thesis explains the $\pi^{\mp}\Sigma^{\pm}$ and the $d(K^-, p)"\pi^-\Sigma^0"$ modes analysis,
so BPD was used only energy loss calculation.
