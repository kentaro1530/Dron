\section{Experimental facility}
\subsection{J-PARC}

The J-PARC is located at the Tokai village in Ibaraki Prefecture.
J-PARC, which means Japan Proton Accelerator Research Complex, consists of some facilities,
which are nuclear transmutation facility, materials, and the Material Life Science Experimental Facility, Neutrino Experimental Facility, and Hadron Experimental Facility\cite{JPARC_had}.
The concept of the J-PARC is to provide various secondary beam for the above purpose.
The J-PARC has three accelerators,
first one is linac which is injector and accelerates proton beam to $400$MeV$/c$,
second is RCS (Rapid Cycling Synchrotrons) which accelerates proton beam to $3$GeV, which was provided to materials and life experimental facility and muon facility.
Next is MR (Main Ring) which accelerates proton beam to $30$GeV$/c$, which beam was extracted by two methods.
One is the fast extraction (FX) for the Neutrino Experimental Facility to produce a neutrino beam which transported to the super Kamiokande.
The other is the slow extraction (SX) for the Hadron Experimental Facility.
In this extraction, banched beam in the MR is gradually extracted as scraping.
For this purpose, the $30$GeV$/c$ proton beam was extracted about 2-second with a 5.2-second repetition cycle in the present experimental.

This continuous beam is irradiated on the primary target that is 6mm $\times$ 6mm $\times$ 66mm golden block
to generate secondary beam which includes anti-proton, pion, kaon and so on that is not naturally exist.
The secondary beam is transported to several beamlines.

The present experiment ix performed at the K1.8BR beamline, which was placed at north of the hadron facility and branced from the K1.8 beamline
which is optimized for beam whose momentum is around $1.8$GeV$/c$.
