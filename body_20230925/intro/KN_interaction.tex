\section{$\Lambda(1405)$ and the $\bar{K}N$ interaction}
As mentioned as before section, the $\Lambda(1405)$ has been predicted as a quasi-bound state of the $\bar{K}N$ state and has been discussed as such.
In order words, it is necessary the information of the $\bar{K}N$ interaction in order to understand the structure of the $\Lambda(1405)$.
In the 1960-70's, various $\bar{K}N$ scattering data from $K^-$ beam were measured with the bubble chamber at the CERN \cite{CERN1,CERN2,CERN3,CERN4,Gopal}.
These data were included up to $2.1$GeV in the center-of-mass frame and were fitted by partial wave analysis.
The KSU group reported the results of a partial wave analysis using all available $\bar{K}N$ scattering data \cite{KSU,KSU2}.
The $\bar{K}N$ scattering amplitudes are well analysed, especially in the high energy region.
Kamano et. el. developed an improved method, the dynamically coupled channel (DCC) model with two models depending on the treatment of meson-baryon interactions,
which was so-called model.A and model.B \cite{DCC1}.
A similar picture to the chiral unitary model, where $\Lambda(1405)$ has two poles, was obtained for both models,
although the predictions were based on extrapolating the amplitude below the $\bar{K}N$ threshold.

Also, one method of measuring the $\bar{K}N$ scattering length at the $\bar{K}N$ threshold is X-ray from kaonic nuclei.
In this method, the X-ray shift emitted by the capture of $K^-$ meson by the nuclei is compared with that of the electromagnetic force alone to assess the effect of the strong force.
In 1970-80's, some groups reported $\bar{K}N$ interaction is repulsive by X-ray from the kaonic hydrogen at the CERN, which is inconsistent with the scattering experiment.
In 1997, Iwasaki et al. reported this negative shift from a high-resolution experiment at KEK-PS E228 and concluded that the $\bar{K}N$ interaction was an attractive force \cite{KEK_E228}.
This result was verified and updated by the Dear collaboration in 2005 \cite{DEAR} and the SIDDAHARTA collaboration in 2011 \cite{SIDDAHARTA}.
The $\bar{K}N$ scattering lengths and effective range obtained by these experiments provide strong constraints on the $\bar{K}N$ scattering amplitude at the $\bar{K}N$ threshold.

Some theoretical groups consistently reproduced  the $\bar{K}N$ scattering amplitude on this constraint and scattering data above the $\bar{K}N$ threshold
using various approch based on low energy scattering therm \cite{Ikeda1,Ikeda2,Mai,Guo,Mai2}.


