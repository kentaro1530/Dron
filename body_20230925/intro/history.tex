\section{History of the $\Lambda(1405)$}
The $\Lambda(1405)$ is a hyperon containing strangeness $S=-1$ with isospin $I=0$ and spin and spin-parity $J^P=(\frac{1}{2})^-$.
In the latest particle data group (PDG) \cite{PDG}, the mass and the width of the $\Lambda(1405)$ are assigned to $1405.1^{+1.3}_{-1.0}$MeV and $50.5\pm 2.0$MeV respectively,
based on several articles \cite{Dalitz, HADES_pheno, Esmaili}. 

The existance of the $\Lambda(1405)$ was first predicted by Dalitz and Taun in 1959 as the quasi-bound state of the $\bar{K}N$ \cite{Dalitz_1st}.
At the Lawrence Radiation Laboratory,
a $\Lambda(1405)$-like excess state was observed in 1961 by the bubble chamber in the $\pi\Sigma$ spectrum using the $K^- p\rightarrow\Sigma \pi \pi \pi$ reaction\cite{L1405_LRL}.
They reported a $\Lambda(1405)$-like excess state in the neutral $\pi\Sigma$ spectrum against to the double charged spectrum, for example $\pi^-\Sigma^-$ or $\pi^+ \Sigma^+$ spectra.
Hemingway reported the successful high-statistics production of $\Lambda(1405)$ by a hydrogen bubble chamber using a $4.2$ GeV $K^-$ beam\cite{Hemingway}.
They claimed that the identification of the $K^- p \rightarrow \pi \Sigma(1660) \rightarrow \pi \pi \Lambda(1405) \rightarrow \pi \pi (\pi \Sigma)$ reaction lemma
enhanced the production of the $\Lambda(1405)$.
Dalitz and Deloff applied M-matrix/K-matrix analysis to the $\pi^-\Sigma^+$ spectrum, which is expected to be background-free from non-resonant and $\Lambda(1520)$ in this data,
and evaluated the mass and width of $\Lambda(1405)$ at $1406.4 \pm 4.0$MeV and $50\pm 2$MeV, respectively\cite{Dalitz}.
This data is employed PDG's estimation for the mass and width of $\Lambda(1405)$.

In the 2000's, the chiral unitary model claimed that the $\Lambda(1405)$ is a dynamical molecular state contributed from two poles,
$\pi\Sigma$ in the low-mass region and $\bar{K}N$ in the high-mass region.
According to the this model, the high-mass pole coupled to $\bar{K}N$ is $1426+16$MeV and the low-mass pole coupled to $\pi\Sigma$ is $1390+66i$MeV.
That means the $\pi \Sigma$ spectrum is expected to shift high mass region from the conventional $1405$MeV by directly accessing to the $\bar{K}N$ pole.

Experimentally, the $\Lambda(1405)$ production was also carried out employing various reaction mechanisms.

B.Riley et. el. reported $\Sigma^{\pm}\pi^{\mp}$ invariant mass of
$K^{-} {}^4\mbox{He} \rightarrow \pi^{\pm} \Sigma^{\mp}$ at rest reaction by stopped $K^-$ using the bubble chamber at Argonne National Laboratory.
The analysis by Esmaili et al. based on this experimental results is employed PDG's estimation of the mass of $\Lambda(1405)$ \cite{Esmaili}.

Niiyama et el. performed photoproduction $\gamma p \rightarrow K^+ \Lambda(1405)$ employing a $\gamma$ beam at $E_{\gamma}=1.5-2.4$GeV at the LEPS beamline in the Spring-8\cite{Niiyama}.
They measured the scattering angle in center of mass system of $K^+$ at $0.8<\Theta_{K^+}<1.0$,
reported mass spectra of $\pi^-\Sigma^+$ and $\pi^+\Sigma^-$ in the $\Lambda(1405) \rightarrow \pi \Sigma$ decay
and observed a difference between the two spectra in the $\Lambda(1405)$ region.
This fact means existance of the interference term between the isospin $I=0$ and $I=1$ channel.
The CLAS collaboration employing a $1.61$-$1.91$GeV $\gamma$ beam for photoproduction at the Jefferson Labolatory
and measured the scattering angle in center of mass system of $K^+$ at $0.6<\Theta_{K^+}<0.9$ \cite{CLAS,CLAS2}.
They reported all three $\pi^-\Sigma^+$, $\pi^+\Sigma^-$ and $\Sigma^0\pi^0$ spectra.
The centroid of those three spectra appear to be at $1405$MeV, but their lineshapes are different indicating contribution of $I=1$ strength in this reaction.

The HADES collaboration performed $\Lambda(1405)$ production in p-p collisions using the $3.5\mbox{GeV}/c$ proton beam \cite{HADES}.
They reported $\pi^-\Sigma^+$, $\pi^+\Sigma^-$ and these average spectra, which clearly show a peak below $1400$MeV.
The analysis by Hassanvand et al. based on this experimental results is employed PDG's estimation of the mass and the width of $\Lambda(1405)$ \cite{HADES_pheno}.

Therefore, the spectrum of $\Lambda(1405)$ depends on the reaction mechanism and the $\pi\Sigma$ charge state for $\Lambda(1405) \rightarrow \pi \Sigma$.
This strongly suggests that $\Lambda(1405)$ is a dynamic state, but the mechanism of these reactions is still controversial and its structure is unknown.
