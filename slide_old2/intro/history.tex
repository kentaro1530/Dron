\begin{frame}{History of $\Lambda(1405)$}
  1959 Prediction as $\bar{K}N$ bound state by R. H. Dalitz and S. F. Taun.
  \vspace{2mm} \\
  1961 Candidate was reported at Lawrence Radiation Laboratory. \\
  \begin{itemize}
  \item Using $K^- p \rightarrow \pi^\pm \pi^\mp (\pi^\pm \Sigma^\mp)$ reaction.
  \item Low statistics. Ambiguity of decay chain.
  \end{itemize}
  1985 High stastic experiment by R. J. Hemingway.
  \begin{itemize}
  \item Using $K^- p \rightarrow \pi^- \Sigma^+(1660)^+ \rightarrow \pi^- \pi^+ (\pi^\mp \Sigma^\pm)$ reaction.\\
  \item $t(K^-, \pi^-)<1.0$ GeV$/c$ selection increased the purity of $\Sigma^+(1660)$, espacially $\pi^- \Sigma^+$ mode.
  \end{itemize}
  1991 $\Lambda(1405)$ mass and width was determined by R. H. Dalitz et el.
  \begin{itemize}
  \item Using above $K^- d \rightarrow \pi^- \Sigma^+(1660) \rightarrow \pi^- \pi^+ (\pi^- \Sigma^+)$ reaction.
  \item Obtained values was adopted by Particle Data Groups, at first.
  \end{itemize}

  \centering
  % $\Rightarrow$
  These analysis (values)\\
  are so called {\bf Phenomenological Potencial}.
\end{frame}

