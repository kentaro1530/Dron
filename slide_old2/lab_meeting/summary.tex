\begin{frame}{結論}
  $1.5\sigma$相当でカットしたところで$\chi^2/NDF$がサチレートしたとして\\
  (P.\ref{page:pole},\ref{page:width}参照)
  この値で$\Lambda(1405)$の極と幅のエラーを決定する。

  $\mbox{Pole}=1418.3^{+7.5}_{-2.4}(fit)^{+0.9}_{-1.2}(syst.)+[-27.8^{+9.5}_{-0.9}(fit)^{+1.9}_{-2.2}(syst.)]i$ [MeV]\\
  \sout{$\mbox{Pole}=1418.3^{+7.3}_{-2.2}(fit)^{+1.2}_{-1.0}(syst.)+[-27.8^{+9.4}_{-0.7}(fit)^{+1.9}_{-2.0}(syst.)]i$ [MeV] \\(FWHMでフィット範囲を決める)}\\
  \sout{$\mbox{Pole}=1418.3^{+7.3}_{-2.3}(fit)^{+1.0}_{-1.1}(syst.)+[-27.8^{+9.5}_{-0.8}(fit)^{+1.9}_{-2.0}(syst.)]i$ [MeV] \\(イテレーションフィット)}\\  
  {\bf Noumi's Parameter}\\
  $\mbox{Pole}=1417.7^{+6.0}_{-7.4}(fit)^{+1.1}_{-1.0}(syst.)+[-26.1^{+6.0}_{-7.9}(fit)^{+1.7}_{-2.0}(syst.)]i$ [MeV]
  \vspace{5mm}\\

  To do\\
  \hspace{3mm} D論にまとめる。
\end{frame}
