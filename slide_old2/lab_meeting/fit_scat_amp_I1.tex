\begin{frame}{Fitting for $\pi^- \Sigma^0$ ($I=1$)}
  \begin{tabular}{cc}
    \begin{minipage}{0.4\hsize}
      \begin{figure}
        \includegraphics[width=5cm]{../pic/Dron/fit_scat_amp_I1.eps}
      \end{figure}

      {
        \scriptsize
        青線: Fit結果 (破線は分解能なし) \\
        黒: 反応関数\\
        赤: 2ステップ目の$|T_2|$
      }
      
    \end{minipage}
    \begin{minipage}{0.6\hsize}
      \footnotesize
      1ステップ目の反応関数は野海さんからのデータ\\
      2ステップ目の$A,R$とスケールがパラメーター
      \vspace{3mm}\\
      $Scale=0.0131 \pm 0.002$ \\
      $A=1.12 \pm 1.18 + (1.95 \pm 0.43)i $\\
      $R=-0.16 \pm 1.28 + (-0.18 \pm 0.37)i $
      \vspace{3mm}\\
      $\Rightarrow$ $A_{re}$のエラーが大きい\\
      \hspace{5mm} $I=1$への感度はほぼ無い\\
      $\Rightarrow$ $A_{re}$はカイラル解析(次ページ)では$0.45\sim 0.6$\\
      \hspace{5mm} 今までの理論解析とは合わない
      \hspace{5mm} 
    \end{minipage}
  \end{tabular}
\end{frame}
