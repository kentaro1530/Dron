\begin{frame}{Fitting for $\pi^- \Sigma^0$ ($I=1$)のまとめと今後}
  \begin{itemize}
  \item $\bar{K}N$閾値以下への染み出しはカスプの効果でありそう。\\
    %   $\Rightarrow$ 分解能でナマされていて確定的には言えない。
  \item 散乱長(の実部)のエラーが値より大きい。\\
    $\rightarrow$ 確定的なことは言えない。\\
  \item 散乱長の実部の値、1.12はカイラルの理論解析の値、\\$.45\sim0.6$より外れている。\\
    $\Rightarrow$ このフィッティングを正しいと信じて議論はしづらい。
  \end{itemize}
  
  {\bf 改善案}
  \begin{itemize}
  \item 荷電モード、$I=0$,$I=1$と干渉項を加えてフィッティング\\
    $I=0$の5つのパラメーターと相対位相パラメーターが増える。\\
    $\rightarrow$ $A_{I=0}$, $R_{I=0}$は野海解析で固定\\
    \hspace{5mm} $I=1$の5個、$I=0$のスケール、相対位相の7つのパラメーター
  \end{itemize}
%   \vspace{3mm}\\

  この解析の結果はわからない\\
  $\Rightarrow$ D論は今まで(理論計算、主にDCC計算との比較)で$I=1$\\と干渉項の重要性を議論する。
\end{frame}
